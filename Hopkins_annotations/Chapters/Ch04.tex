Jonathan Harker's Journal Continued 

I awoke in my own bed. If it be that I had not dreamt, the Count must have carried me here. I tried to satisfy myself on the subject, but could not arrive at any unquestionable result. To be sure, there were certain small evidences, such as that my clothes were folded and laid by in a manner which was not my habit. My watch was still unwound, and I am rigorously accustomed to wind it the last thing before going to bed, and many such details. But these things are no proof, for they may have been evidences that my mind was not as usual, and, for some cause or another, I had certainly been much upset. I must watch for proof. Of one thing I am glad. If it was that the Count carried me here and undressed me, he must have been hurried in his task, for my pockets are intact. I am sure this diary would have been a mystery to him which he would not have brooked. He would have taken or destroyed it. As I look round this room, although it has been to me so full of fear, it is now a sort of sanctuary, for nothing can be more dreadful than those awful women, who were, who are, waiting to suck my blood. 

18 May.--I have been down to look at that room again in daylight, for I must know the truth. When I got to the doorway at the top of the stairs I found it closed. It had been so forcibly driven against the jamb that part of the woodwork was splintered. I could see that the bolt of the lock had not been shot, but the door is fastened from the inside. I fear it was no dream, and must act on this surmise. 

19 May.--I am surely in the toils. Last night the Count asked me in the sauvest tones to write three letters, one saying that my work here was nearly done, and that I should start for home within a few days, another that I was starting on the next morning from the time of the letter, and the third that I had left the castle and arrived at Bistritz. I would fain have rebelled, but felt that in the present state of things it would be madness to quarrel openly with the Count whilst I am so absolutely in his power. And to refuse would be to excite his suspicion and to arouse his anger. He knows that I know too much, and that I must not live, lest I be dangerous to him. My only chance is to prolong my opportunities. Something may occur which will give ma a chance to escape. I saw in his eyes something of that gathering wrath which was manifest when he hurled that fair woman from him. He explained to me that posts were few and uncertain, and that my writing now would ensure ease of mind to my friends. And he assured me with so much impressiveness that he would countermand the later letters, which would be held over at Bistritz until due time in case chance would admit of my prolonging my stay, that to oppose him would have been to create new suspicion. I therefore pretended to fall in with his views, and asked him what dates I should put on the letters. 

He calculated a minute, and then said, "The first should be June 12,the second June 19,and the third June 29." 

I know now the span of my life. God help me! 

28 May.--There is a chance of escape, or at any rate of being able to send word home. A band of Szgany have come to the castle, and are encamped in the courtyard. These are gipsies. I have notes of them in my book. They are peculiar to this part of the world, though allied to the ordinary gipsies all the world over. There are thousands of them in Hungary and Transylvania, who are almost outside all law. They attach themselves as a rule to some great noble or boyar, and call themselves by his name. They are fearless and without religion, save superstition, and they talk only their own varieties of the Romany tongue. 

I shall write some letters home, and shall try to get them to have them posted. I have already spoken to them through my window to begin acquaintanceship. They took their hats off and made obeisance and many signs, which however, I could not understand any more than I could their spoken language . . . 

I have written the letters. Mina's is in shorthand, and I simply ask Mr. Hawkins to communicate with her. To her I have explained my situation, but without the horrors which I may only surmise. It would shock and frighten her to death were I to expose my heart to her. Should the letters not carry, then the Count shall not yet know my secret or the extent of my knowledge . . . 

I have given the letters. I threw them through the bars of my window with a gold piece, and made what signs I could to have them posted. The man who took them pressed them to his heart and bowed, and then put them in his cap. I could do no more. I stole back to the study, and began to read. As the Count did not come in, I have written here . . . 

The Count has come. He sat down beside me, and said in his smoothest voice as he opened two letters, "The Szgany has given me these, of which, though I know not whence they come, I shall, of course, take care. See!"--He must have looked at it.--"One is from you, and to my friend Peter Hawkins. The other,"--here he caught sight of the strange symbols as he opened the envelope, and the dark look came into his face, and his eyes blazed wickedly,--"The other is a vile thing, an outrage upon friendship and hospitality! It is not signed. Well! So it cannot matter to us."And he calmly held letter and envelope in the flame of the lamp till they were consumed. 

Then he went on, "The letter to Hawkins, that I shall, of course send on, since it is yours. Your letters are sacred to me. Your pardon, my friend, that unknowingly I did break the seal. Will you not cover it again?"He held out the letter to me, and with a courteous bow handed me a clean envelope. 

I could only redirect it and hand it to him in silence. When he went out of the room I could hear the key turn softly. A minute later I went over and tried it, and the door was locked. 

When, an hour or two after, the Count came quietly into the room, his coming awakened me, for I had gone to sleep on the sofa. He was very courteous and very cheery in his manner, and seeing that I had been sleeping, he said, "So, my friend, you are tired? Get to bed. There is the surest rest. I may not have the pleasure of talk tonight, since there are many labours to me, but you will sleep, I pray." 

I passed to my room and went to bed, and, strange to say, slept without dreaming. Despair has its own calms. 

31 May.--This morning when I woke I thought I would provide myself with some papers and envelopes from my bag and keep them in my pocket, so that I might write in case I should get an opportunity, but again a surprise, again a shock! 

Every scrap of paper was gone, and with it all my notes, my memoranda, relating to railways and travel, my letter of credit, in fact all that might be useful to me were I once outside the castle. I sat and pondered awhile, and then some thought occurred to me, and I made search of my portmanteau and in the wardrobe where I had placed my clothes. 

The suit in which I had travelled was gone, and also my overcoat and rug. I could find no trace of them anywhere. This looked like some new scheme of villainy . . . 

17 June.--This morning, as I was sitting on the edge of my bed cudgelling my brains, I heard without a crackling of whips and pounding and scraping of horses' feet up the rocky path beyond the courtyard. With joy I hurried to the window, and saw drive into the yard two great leiter-wagons, each drawn by eight sturdy horses, and at the head of each pair a Slovak, with his wide hat, great nail-studded belt, dirty sheepskin, and high boots. They had also their long staves in hand. I ran to the door, intending to descend and try and join them through the main hall, as I thought that way might be opened for them. Again a shock, my door was fastened on the outside. 

Then I ran to the window and cried to them. They looked up at me stupidly and pointed, but just then the "hetman" of the Szgany came out, and seeing them pointing to my window, said something, at which they laughed. 

Henceforth no effort of mine, no piteous cry or agonized entreaty, would make them even look at me. They resolutely turned away. The leiter-wagons contained great, square boxes, with handles of thick rope. These were evidently empty by the ease with which the Slovaks handled them, and by their resonance as they were roughly moved. 

When they were all unloaded and packed in a great heap in one corner of the yard, the Slovaks were given some money by the Szgany, and spitting on it for luck, lazily went each to his horse's head. Shortly afterwards, I heard the crackling of their whips die away in the distance. 

24 June.--Last night the Count left me early, and locked himself into his own room. As soon as I dared I ran up the winding stair, and looked out of the window, which opened South. I thought I would watch for the Count, for there is something going on. The Szgany are quartered somewhere in the castle and are doing work of some kind. I know it, for now and then, I hear a far-away muffled sound as of mattock and spade, and, whatever it is, it must be the end of some ruthless villainy. 

I had been at the window somewhat less than half an hour, when I saw something coming out of the Count's window. I drew back and watched carefully, and saw the whole man emerge. It was a new shock to me to find that he had on the suit of clothes which I had worn whilst travelling here, and slung over his shoulder the terrible bag which I had seen the women take away. There could be no doubt as to his quest, and in my garb, too! This, then, is his new scheme of evil, that he will allow others to see me, as they think, so that he may both leave evidence that I have been seen in the towns or villages posting my own letters, and that any wickedness which he may do shall by the local people be attributed to me. 

It makes me rage to think that this can go on, and whilst I am shut up here, a veritable prisoner, but without that protection of the law which is even a criminal's right and consolation. 

I thought I would watch for the Count's return, and for a long time sat doggedly at the window. Then I began to notice that there were some quaint little specks floating in the rays of the moonlight. They were like the tiniest grains of dust, and they whirled round and gathered in clusters in a nebulous sort of way. I watched them with a sense of soothing, and a sort of calm stole over me. I leaned back in the embrasure in a more comfortable position, so that I could enjoy more fully the aerial gambolling. 

Something made me start up, a low, piteous howling of dogs somewhere far below in the valley, which was hidden from my sight. Louder it seemed to ring in my ears, and the floating moats of dust to take new shapes to the sound as they danced in the moonlight. I felt myself struggling to awake to some call of my instincts. Nay, my very soul was struggling, and my half-remembered sensibilities were striving to answer the call. I was becoming hypnotised! 

Quicker and quicker danced the dust. The moonbeams seemed to quiver as they went by me into the mass of gloom beyond. More and more they gathered till they seemed to take dim phantom shapes. And then I started, broad awake and in full possession of my senses, and ran screaming from the place. 

The phantom shapes, which were becoming gradually materialised from the moonbeams, were those three ghostly women to whom I was doomed. 

I fled, and felt somewhat safer in my own room, where there was no moonlight, and where the lamp was burning brightly. 

When a couple of hours had passed I heard something stirring in the Count's room, something like a sharp wail quickly suppressed. And then there was silence, deep, awful silence, which chilled me. With a beating heart, I tried the door, but I was locked in my prison, and could do nothing. I sat down and simply cried. 

As I sat I heard a sound in the courtyard without, the agonised cry of a woman. I rushed to the window, and throwing it up, peered between the bars. 

There, indeed, was a woman with dishevelled hair, holding her hands over her heart as one distressed with running. She was leaning against the corner of the gateway. When she saw my face at the window she threw herself forward, and shouted in a voice laden with menace, "Monster, give me my child!" 

She threw herself on her knees, and raising up her hands, cried the same words in tones which wrung my heart. Then she tore her hair and beat her breast, and abandoned herself to all the violences of extravagant emotion. Finally, she threw herself forward, and though I could not see her, I could hear the beating of her naked hands against the door. 

Somewhere high overhead, probably on the tower, I heard the voice of the Count calling in his harsh, metallic whisper. His call seemed to be answered from far and wide by the howling of wolves. Before many minutes had passed a pack of them poured, like a pent-up dam when liberated, through the wide entrance into the courtyard. 

There was no cry from the woman, and the howling of the wolves was but short. Before long they streamed away singly, licking their lips. 

I could not pity her, for I knew now what had become of her child, and she was better dead. 

What shall I do? What can I do? How can I escape from this dreadful thing of night, gloom, and fear? 

25 June.--No man knows till he has suffered from the night how sweet and dear to his heart and eye the morning can be. When the sun grew so high this morning that it struck the top of the great gateway opposite my window, the high spot which it touched seemed to me as if the dove from the ark had lighted there. My fear fell from me as if it had been a vaporous garment which dissolved in the warmth. 

I must take action of some sort whilst the courage of the day is upon me. Last night one of my post-dated letters went to post, the first of that fatal series which is to blot out the very traces of my existence from the earth. 

Let me not think of it. Action! 

It has always been at night-time that I have been molested or threatened, or in some way in danger or in fear. I have not yet seen the Count in the daylight. Can it be that he sleeps when others wake, that he may be awake whilst they sleep? If I could only get into his room! But there is no possible way. The door is always locked, no way for me. 

Yes, there is a way, if one dares to take it. Where his body has gone why may not another body go? I have seen him myself crawl from his window. Why should not I imitate him, and go in by his window? The chances are desperate, but my need is more desperate still. I shall risk it. At the worst it can only be death, and a man's death is not a calf's, and the dreaded Hereafter may still be open to me. God help me in my task! Goodbye, Mina, if I fail. Goodbye, my faithful friend and second father. Goodbye, all, and last of all Mina! 

Same day, later.--I have made the effort, and God helping me, have come safely back to this room. I must put down every detail in order. I went whilst my courage was fresh straight to the window on the south side, and at once got outside on this side. The stones are big and roughly cut, and the mortar has by process of time been washed away between them. I took off my boots, and ventured out on the desperate way. I looked down once, so as to make sure that a sudden glimpse of the awful depth would not overcome me, but after that kept my eyes away from it. I know pretty well the direction and distance of the Count's window, and made for it as well as I could, having regard to the opportunities available. I did not feel dizzy, I suppose I was too excited, and the time seemed ridiculously short till I found myself standing on the window sill and trying to raise up the sash. I was filled with agitation, however, when I bent down and slid feet foremost in through the window. Then I looked around for the Count, but with surprise and gladness, made a discovery. The room was empty! It was barely furnished with odd things, which seemed to have never been used. 

The furniture was something the same style as that in the south rooms, and was covered with dust. I looked for the key, but it was not in the lock, and I could not find it anywhere. The only thing I found was a great heap of gold in one corner, gold of all kinds, Roman, and British, and Austrian, and Hungarian,and Greek and Turkish money, covered with a film of dust, as though it had lain long in the ground. None of it that I noticed was less than three hundred years old. There were also chains and ornaments, some jewelled, but all of them old and stained. 

At one corner of the room was a heavy door. I tried it, for, since I could not find the key of the room or the key of the outer door, which was the main object of my search, I must make further examination, or all my efforts would be in vain. It was open, and led through a stone passage to a circular stairway, which went steeply down. 

I descended, minding carefully where I went for the stairs were dark, being only lit by loopholes in the heavy masonry. At the bottom there was a dark, tunnel-like passage, through which came a deathly, sickly odour, the odour of old earth newly turned. As I went through the passage the smell grew closer and heavier. At last I pulled open a heavy door which stood ajar, and found myself in an old ruined chapel, which had evidently been used as a graveyard. The roof was broken, and in two places were steps leading to vaults, but the ground had recently been dug over, and the earth placed in great wooden boxes, manifestly those which had been brought by the Slovaks. 

There was nobody about, and I made a search over every inch of the ground, so as not to lose a chance. I went down even into the vaults, where the dim light struggled, although to do so was a dread to my very soul. Into two of these I went, but saw nothing except fragments of old coffins and piles of dust. In the third, however, I made a discovery. 

There, in one of the great boxes, of which there were fifty in all, on a pile of newly dug earth, lay the Count! He was either dead or asleep. I could not say which, for eyes were open and stony, but without the glassiness of death, and the cheeks had the warmth of life through all their pallor. The lips were as red as ever. But there was no sign of movement, no pulse, no breath, no beating of the heart. 

I bent over him, and tried to find any sign of life, but in vain. He could not have lain there long, for the earthy smell would have passed away in a few hours. By the side of the box was its cover, pierced with holes here and there. I thought he might have the keys on him, but when I went to search I saw the dead eyes, and in them dead though they were, such a look of hate, though unconscious of me or my presence, that I fled from the place, and leaving the Count's room by the window, crawled again up the castle wall. Regaining my room, I threw myself panting upon the bed and tried to think. 

29 June.--Today is the date of my last letter, and the Count has taken steps to prove that it was genuine, for again I saw him leave the castle by the same window, and in my clothes. As he went down the wall, lizard fashion, I wished I had a gun or some lethal weapon, that I might destroy him. But I fear that no weapon wrought along by man's hand would have any effect on him. I dared not wait to see him return, for I feared to see those weird sisters. I came back to the library, and read there till I fell asleep. 

I was awakened by the Count, who looked at me as grimly as a man could look as he said,"Tomorrow, my friend, we must part. You return to your beautiful England, I to some work which may have such an end that we may never meet. Your letter home has been despatched. Tomorrow I shall not be here, but all shall be ready for your journey. In the morning come the Szgany, who have some labours of their own here, and also come some Slovaks. When they have gone, my carriage shall come for you, and shall bear you to the Borgo Pass to meet the diligence from Bukovina to Bistritz. But I am in hopes that I shall see more of you at Castle Dracula." 

I suspected him, and determined to test his sincerity. Sincerity! It seems like a profanation of the word to write it in connection with such a monster, so I asked him pointblank, "Why may I not go tonight?" 

"Because, dear sir, my coachman and horses are away on a mission." 

"But I would walk with pleasure. I want to get away at once." 

He smiled, such a soft, smooth, diabolical smile that I knew there was some trick behind his smoothness. He said, "And your baggage?" 

"I do not care about it. I can send for it some other time." 

The Count stood up, and said, with a sweet courtesy which made me rub my eyes, it seemed so real, "You English have a saying which is close to my heart, for its spirit is that which rules our boyars, `Welcome the coming, speed the parting guest.' Come with me, my dear young friend. Not an hour shall you wait in my house against your will, though sad am I at your going, and that you so suddenly desire it. Come!" With a stately gravity, he, with the lamp, preceded me down the stairs and along the hall. Suddenly he stopped. "Hark!" 

Close at hand came the howling of many wolves. It was almost as if the sound sprang up at the rising of his hand, just as the music of a great orchestra seems to leap under the baton of the conductor. After a pause of a moment, he proceeded, in his stately way, to the door, drew back the ponderous bolts, unhooked the heavy chains, and began to draw it open. 

To my intense astonishment I saw that it was unlocked. Suspiciously, I looked all round, but could see no key of any kind. 

As the door began to open, the howling of the wolves without grew louder and angrier. Their red jaws, with champing teeth, and their blunt-clawed feet as they leaped, came in through the opening door. I knew than that to struggle at the moment against the Count was useless. With such allies as these at his command, I could do nothing. 

But still the door continued slowly to open, and only the Count's body stood in the gap. Suddenly it struck me that this might be the moment and means of my doom. I was to be given to the wolves, and at my own instigation. There was a diabolical wickedness in the idea great enough for the Count, and as the last chance I cried out, "Shut the door! I shall wait till morning." And I covered my face with my hands to hide my tears of bitter disappointment. 

With one sweep of his powerful arm, the Count threw the door shut, and the great bolts clanged and echoed through the hall as they shot back into their places. 

In silence we returned to the library, and after a minute or two I went to my own room. The last I saw of Count Dracula was his kissing his hand to me, with a red light of triumph in his eyes, and with a smile that Judas in hell might be proud of. 

When I was in my room and about to lie down, I thought I heard a whispering at my door. I went to it softly and listened. Unless my ears deceived me, I heard the voice of the Count. 

"Back! Back to your own place! Your time is not yet come. Wait! Have patience! Tonight is mine. Tomorrow night is yours!" 

There was a low, sweet ripple of laughter, and in a rage I threw open the door, and saw without the three terrible women licking their lips. As I appeared, they all joined in a horrible laugh, and ran away. 

I came back to my room and threw myself on my knees. It is then so near the end? Tomorrow! Tomorrow! Lord, help me, and those to whom I am dear! 

30 June.--These may be the last words I ever write in this diary. I slept till just before the dawn, and when I woke threw myself on my knees, for I determined that if Death came he should find me ready. 

At last I felt that subtle change in the air, and knew that the morning had come. Then came the welcome cock-crow, and I felt that I was safe. With a glad heart, I opened the door and ran down the hall. I had seen that the door was unlocked, and now escape was before me. With hands that trembled with eagerness, I unhooked the chains and threw back the massive bolts. 

But the door would not move. Despair seized me. I pulled and pulled at the door, and shook it till, massive as it was, it rattled in its casement. I could see the bolt shot. It had been locked after I left the Count. 

Then a wild desire took me to obtain the key at any risk, and I determined then and there to scale the wall again, and gain the Count's room. He might kill me, but death now seemed the happier choice of evils. Without a pause I rushed up to the east window, and scrambled down the wall, as before, into the Count's room. It was empty, but that was as I expected. I could not see a key anywhere, but the heap of gold remained. I went through the door in the corner and down the winding stair and along the dark passage to the old chapel. I knew now well enough where to find the monster I sought. 

The great box was in the same place, close against the wall, but the lid was laid on it, not fastened down, but with the nails ready in their places to be hammered home. 

I knew I must reach the body for the key, so I raised the lid, and laid it back against the wall. And then I saw something which filled my very soul with horror. There lay the Count, but looking as if his youth had been half restored. For the white hair and moustache were changed to dark irongrey. The cheeks were fuller, and the white skin seemed ruby-red underneath. The mouth was redder than ever, for on the lips were gouts of fresh blood, which trickled from the corners of the mouth and ran down over the chin and neck. Even the deep, burning eyes seemed set amongst swollen flesh, for the lids and pouches underneath were bloated. It seemed as if the whole awful creature were simply gorged with blood. He lay like a filthy leech, exhausted with his repletion. 

I shuddered as I bent over to touch him, and every sense in me revolted at the contact, but I had to search, or I was lost. The coming night might see my own body a banquet in a similar war to those horrid three. I felt all over the body, but no sign could I find of the key. Then I stopped and looked at the Count. There was a mocking smile on the bloated face which seemed to drive me mad. This was the being I was helping to transfer to London, where, perhaps, for centuries to come he might, amongst its teeming millions, satiate his lust for blood, and create a new and ever-widening circle of semi-demons to batten on the helpless. 

The very thought drove me mad. A terrible desire came upon me to rid the world of such a monster. There was no lethal weapon at hand, but I seized a shovel which the workmen had been using to fill the cases, and lifting it high, struck, with the edge downward, at the hateful face. But as I did so the head turned, and the eyes fell upon me, with all their blaze of basilisk horror. The sight seemed to paralyze me, and the shovel turned in my hand and glanced from the face, merely making a deep gash above the forehead. The shovel fell from my hand across the box, and as I pulled it away the flange of the blade caught the edge of the lid which fell over again, and hid the horrid thing from my sight. The last glimpse I had was of the bloated face, blood-stained and fixed with a grin of malice which would have held its own in the nethermost hell. I thought and thought what should be my next move, but my brain seemed on fire, and I waited with a despairing feeling growing over me. As I waited I heard in the distance a gipsy song sung by merry voices coming closer, and through their song the rolling of heavy wheels and the cracking of whips. The Szgany and the Slovaks of whom the Count had spoken were coming. With a last look around and at the box which contained the vile body, I ran from the place and gained the Count's room, determined to rush out at the moment the door should be opened. With strained ears, I listened, and heard downstairs the grinding of the key in the great lock and the falling back of the heavy door. There must have been some other means of entry, or some one had a key for one of the locked doors. 

Then there came the sound of many feet tramping and dying away in some passage which sent up a clanging echo. I turned to run down again towards the vault, where I might find the new entrance, but at the moment there seemed to come a violent puff of wind, and the door to the winding stair blew to with a shock that set the dust from the lintels flying. When I ran to push it open, I found that it was hopelessly fast. I was again a prisoner, and the net of doom was closing round me more closely. 

As I write there is in the passage below a sound of many tramping feet and the crash of weights being set down heavily, doubtless the boxes, with their freight of earth. There was a sound of hammering. It is the box being nailed down. Now I can hear the heavy feet tramping again along the hall, with with many other idle feet coming behind them. 

The door is shut, the chains rattle. There is a grinding of the key in the lock. I can hear the key withdrawn, then another door opens and shuts. I hear the creaking of lock and bolt. 

Hark! In the courtyard and down the rocky way the roll of heavy wheels, the crack of whips, and the chorus of the Szgany as they pass into the distance. 

I am alone in the castle with those horrible women. Faugh! Mina is a woman, and there is nought in common. They are devils of the Pit! 

I shall not remain alone with them. I shall try to scale the castle wall farther than I have yet attempted. I shall take some of the gold with me, lest I want it later. I may find a way from this dreadful place. 

And then away for home! Away to the quickest and nearest train! Away from the cursed spot, from this cursed land, where the devil and his children still walk with earthly feet! 

At least God's mercy is better than that of those monsters, and the precipice is steep and high. At its foot a man may sleep, as a man. Goodbye, all. Mina! 
