\chapter{DR. SEWARD'S DIARY--cont}

The funeral was arranged for the next succeeding day, so that Lucy and her mother might be buried together. I attended to all the ghastly formalities, and the urbane undertaker proved that his staff was afflicted, or blessed, with something of his own obsequious suavity. Even the woman who performed the last offices for the dead remarked to me, in a confidential, brother-professional way, when she had come out from the death chamber, 

"She makes a very beautiful corpse, sir. It's quite a privilege to attend on her. It's not too much to say that she will do credit to our establishment!" 

I noticed that Van Helsing never kept far away. This was possible from the disordered state of things in the household. There were no relatives at hand, and as Arthur had to be back the next day to attend at his father's funeral, we were unable to notify any one who should have been bidden. Under the circumstances, Van Helsing and I took it upon ourselves to examine papers, etc. He insisted upon looking over Lucy's papers himself. I asked him why, for I feared that he, being a foreigner, might not be quite aware of English legal requirements, and so might in ignorance make some unnecessary trouble. 

He answered me, "I know, I know. You forget that I am a lawyer as well as a doctor. But this is not altogether for the law. You knew that, when you avoided the coroner. I have more than him to avoid. There may be papers more, such as this." 

As he spoke he took from his pocket book the memorandum which had been in Lucy's breast, and which she had torn in her sleep. 

"When you find anything of the solicitor who is for the late Mrs. Westenra, seal all her papers, and write him tonight. For me, I watch here in the room and in Miss Lucy's old room all night, and I myself search for what may be. It is not well that her very thoughts go into the hands of strangers." 

I went on with my part of the work, and in another half hour had found the name and address of Mrs. Westenra's solicitor and had written to him. All the poor lady's papers were in order. Explicit directions regarding the place of burial were given. I had hardly sealed the letter, when, to my surprise, Van Helsing walked into the room, saying, 

"Can I help you friend John? I am free, and if I may, my service is to you." 

"Have you got what you looked for?" I asked. 

To which he replied, "I did not look for any specific thing. I only hoped to find, and find I have, all that there was, only some letters and a few memoranda, and a diary new begun. But I have them here, and we shall for the present say nothing of them. I shall see that poor lad tomorrow evening, and, with his sanction, I shall use some." 

When we had finished the work in hand, he said to me, "And now, friend John, I think we may to bed. We want sleep, both you and I, and rest to recuperate. Tomorrow we shall have much to do, but for the tonight there is no need of us. Alas!" 

Before turning in we went to look at poor Lucy. The undertaker had certainly done his work well, for the room was turned into a small chapelle ardente. There was a wilderness of beautiful white flowers, and death was made as little repulsive as might be. The end of the winding sheet was laid over the face. When the Professor bent over and turned it gently back, we both started at the beauty before us. The tall wax candles showing a sufficient light to note it well. All Lucy's loveliness had come back to her in death, and the hours that had passed, instead of leaving traces of `decay's effacing fingers', had but restored the beauty of life, till positively I could not believe my eyes that I was looking at a corpse. 

The Professor looked sternly grave. He had not loved her as I had, and there was no need for tears in his eyes. He said to me, "Remain till I return," and left the room. He came back with a handful of wild garlic from the box waiting in the hall, but which had not been opened, and placed the flowers amongst the others on and around the bed. Then he took from his neck, inside his collar, a little gold crucifix, and placed it over the mouth. He restored the sheet to its place, and we came away. 

I was undressing in my own room, when, with a premonitory tap at the door, he entered, and at once began to speak. 

"Tomorrow I want you to bring me, before night, a set of post-mortem knives." 

"Must we make an autopsy?" I asked. 

"Yes and no. I want to operate, but not what you think. Let me tell you now, but not a word to another. I want to cut off her head and take out her heart. Ah! You a surgeon, and so shocked! You, whom I have seen with no tremble of hand or heart, do operations of life and death that make the rest shudder. Oh, but I must not forget, my dear friend John, that you loved her, and I have not forgotten it for is I that shall operate, and you must not help. I would like to do it tonight, but for Arthur I must not. He will be free after his father's funeral tomorrow, and he will want to see her, to see it. Then, when she is coffined ready for the next day, you and I shall come when all sleep. We shall unscrew the coffin lid, and shall do our operation, and then replace all, so that none know, save we alone." 

"But why do it at all? The girl is dead. Why mutilate her poor body without need? And if there is no necessity for a post-mortem and nothing to gain by it, no good to her, to us, to science, to human knowledge, why do it? Without such it is monstrous." 

For answer he put his hand on my shoulder, and said, with infinite tenderness, "Friend John, I pity your poor bleeding heart, and I love you the more because it does so bleed. If I could, I would take on myself the burden that you do bear. But there are things that you know not, but that you shall know, and bless me for knowing, though they are not pleasant things. John, my child, you have been my friend now many years, and yet did you ever know me to do any without good cause? I may err, I am but man, but I believe in all I do. Was it not for these causes that you send for me when the great trouble came? Yes! Were you not amazed, nay horrified, when I would not let Arthur kiss his love, though she was dying, and snatched him away by all my strength? Yes! And yet you saw how she thanked me, with her so beautiful dying eyes, her voice, too, so weak, and she kiss my rough old hand and bless me? Yes! And did you not hear me swear promise to her, that so she closed her eyes grateful? Yes! 

"Well, I have good reason now for all I want to do. You have for many years trust me. You have believe me weeks past, when there be things so strange that you might have well doubt. Believe me yet a little, friend John. If you trust me not, then I must tell what I think, and that is not perhaps well. And if I work, as work I shall, no matter trust or no trust, without my friend trust in me, I work with heavy heart and feel, oh so lonely when I want all help and courage that may be!" He paused a moment and went on solemnly, "Friend John, there are strange and terrible days before us. Let us not be two, but one, that so we work to a good end. Will you not have faith in me?" 

I took his hand, and promised him. I held my door open as he went away, and watched him go to his room and close the door. As I stood without moving, I saw one of the maids pass silently along the passage, she had her back to me, so did not see me, and go into the room where Lucy lay. The sight touched me. Devotion is so rare, and we are so grateful to those who show it unasked to those we love. Here was a poor girl putting aside the terrors which she naturally had of death to go watch alone by the bier of the mistress whom she loved, so that the poor clay might not be lonely till laid to eternal rest. 

I must have slept long and soundly, for it was broad daylight when Van Helsing waked me by coming into my room. He came over to my bedside and said, "You need not trouble about the knives. We shall not do it." 

"Why not?" I asked. For his solemnity of the night before had greatly impressed me. 

"Because," he said sternly, "it is too late, or too early. See!" Here he held up the little golden crucifix. 

"This was stolen in the night." 

"How stolen,"I asked in wonder,"since you have it now?" 

"Because I get it back from the worthless wretch who stole it, from the woman who robbed the dead and the living. Her punishment will surely come, but not through me. She knew not altogether what she did, and thus unknowing, she only stole. Now we must wait." He went away on the word, leaving me with a new mystery to think of, a new puzzle to grapple with. 

The forenoon was a dreary time, but at noon the solicitor came, Mr. Marquand, of Wholeman, Sons, Marquand \& Lidderdale. He was very genial and very appreciative of what we had done, and took off our hands all cares as to details. During lunch he told us that Mrs. Westenra had for some time expected sudden death from her heart, and had put her affairs in absolute order. He informed us that, with the exception of a certain entailed property of Lucy's father which now, in default of direct issue, went back to a distant branch of the family, the whole estate, real and personal, was left absolutely to Arthur Holmwood. When he had told us so much he went on, 

"Frankly we did our best to prevent such a testamentary disposition, and pointed out certain contingencies that might leave her daughter either penniless or not so free as she should be to act regarding a matrimonial alliance. Indeed, we pressed the matter so far that we almost came into collision, for she asked us if we were or were not prepared to carry out her wishes. Of course, we had then no alternative but to accept. We were right in principle, and ninety-nine times out of a hundred we should have proved, by the logic of events, the accuracy of our judgment. 

"Frankly, however, I must admit that in this case any other form of disposition would have rendered impossible the carrying out of her wishes. For by her predeceasing her daughter the latter would have come into possession of the property, and, even had she only survived her mother by five minutes, her property would, in case there were no will, and a will was a practical impossibility in such a case, have been treated at her decease as under intestacy. In which case Lord Godalming, though so dear a friend, would have had no claim in the world. And the inheritors, being remote, would not be likely to abandon their just rights, for sentimental reasons regarding an entire stranger. I assure you, my dear sirs, I am rejoiced at the result,perfectly rejoiced." 

He was a good fellow, but his rejoicing at the one little part, in which he was officially interested, of so great a tragedy, was an object-lesson in the limitations of sympathetic understanding. 

He did not remain long, but said he would look in later in the day and see Lord Godalming. His coming, however, had been a certain comfort to us, since it assured us that we should not have to dread hostile criticism as to any of our acts. Arthur was expected at five o'clock, so a little before that time we visited the death chamber. It was so in very truth, for now both mother and daughter lay in it. The undertaker, true to his craft, had made the best display he could of his goods, and there was a mortuary air about the place that lowered our spirits at once. 

Van Helsing ordered the former arrangement to be adhered to, explaining that, as Lord Godalming was coming very soon, it would be less harrowing to his feelings to see all that was left of his fiancee quite alone. 

The undertaker seemed shocked at his own stupidity and exerted himself to restore things to the condition in which we left them the night before, so that when Arthur came such shocks to his feelings as we could avoid were saved. 

Poor fellow! He looked desperately sad and broken. Even his stalwart manhood seemed to have shrunk somewhat under the strain of his much-tried emotions. He had, I knew, been very genuinely and devotedly attached to his father, and to lose him, and at such a time, was a bitter blow to him. With me he was warm as ever, and to Van Helsing he was sweetly courteous. But I could not help seeing that there was some constraint with him. The professor noticed it too, and motioned me to bring him upstairs. I did so, and left him at the door of the room, as I felt he would like to be quite alone with her, but he took my arm and led me in, saying huskily, 

"You loved her too, old fellow. She told me all about it, and there was no friend had a closer place in her heart than you. I don't know how to thank you for all you have done for her. I can't think yet . . ." 

Here he suddenly broke down, and threw his arms round my shoulders and laid his head on my breast, crying, "Oh, Jack! Jack! What shall I do? The whole of life seems gone from me all at once, and there is nothing in the wide world for me to live for." 

I comforted him as well as I could. In such cases men do not need much expression. A grip of the hand, the tightening of an arm over the shoulder, a sob in unison, are expressions of sympathy dear to a man's heart. I stood still and silent till his sobs died away, and then I said softly to him, "Come and look at her." 

Together we moved over to the bed, and I lifted the lawn from her face. God! How beautiful she was. Every hour seemed to be enhancing her loveliness. It frightened and amazed me somewhat. And as for Arthur, he fell to trembling, and finally was shaken with doubt as with an ague. At last, after a long pause, he said to me in a faint whisper,"Jack, is she really dead?" 

I assured him sadly that it was so, and went on to suggest, for I felt that such a horrible doubt should not have life for a moment longer than I could help, that it often happened that after death faces become softened and even resolved into their youthful beauty, that this was especially so when death had been preceded by any acute or prolonged suffering. I seemed to quite do away with any doubt, and after kneeling beside the couch for a while and looking at her lovingly and long, he turned aside. I told him that that must be goodbye, as the coffin had to be prepared, so he went back and took her dead hand in his and kissed it, and bent over and kissed her forehead. He came away, fondly looking back over his shoulder at her as he came. 

I left him in the drawing room, and told Van Helsing that he had said goodbye, so the latter went to the kitchen to tell the undertaker's men to proceed with the preperations and to screw up the coffin. When he came out of the room again I told him of Arthur's question, and he replied, "I am not surprised. Just now I doubted for a moment myself!" 

We all dined together, and I could see that poor Art was trying to make the best of things. Van Helsing had been silent all dinner time, but when we had lit our cigars he said, "Lord . . ., but Arthur interrupted him. 

"No, no, not that, for God's sake! Not yet at any rate. Forgive me, sir. I did not mean to speak offensively. It is only because my loss is so recent." 

The Professor answered very sweetly, "I only used that name because I was in doubt. I must not call you `Mr.' and I have grown to love you, yes, my dear boy, to love you, as Arthur." 

Arthur held out his hand, and took the old man's warmly. "Call me what you will," he said. "I hope I may always have the title of a friend. And let me say that I am at a loss for words to thank you for your goodness to my poor dear." He paused a moment, and went on, "I know that she understood your goodness even better than I do. And if I was rude or in any way wanting at that time you acted so, you remember,"-- the Professor nodded--"You must forgive me." 

He answered with a grave kindness, "I know it was hard for you to quite trust me then, for to trust such violence needs to understand, and I take it that you do not, that you cannot, trust me now, for you do not yet understand. And there may be more times when I shall want you to trust when you cannot, and may not, and must not yet understand. But the time will come when your trust shall be whole and complete in me, and when you shall understand as though the sunlight himself shone through. Then you shall bless me from first to last for your own sake, and for the sake of others, and for her dear sake to whom I swore to protect." 

"And indeed, indeed, sir," said Arthur warmly. "I shall in all ways trust you. I know and believe you have a very noble heart, and you are Jack's friend, and you were hers. You shall do what you like." 

The Professor cleared his throat a couple of times, as though about to speak, and finally said, "May I ask you something now?" 

"Certainly." 

"You know that Mrs. Westenra left you all her property?" 

"No, poor dear. I never thought of it." 

"And as it is all yours, you have a right to deal with it as you will. I want you to give me permission to read all Miss Lucy's papers and letters. Believe me, it is no idle curiosity. I have a motive of which, be sure, she would have approved. I have them all here. I took them before we knew that all was yours, so that no strange hand might touch them, no strange eye look through words into her soul. I shall keep them, if I may. Even you may not see them yet, but I shall keep them safe. No word shall be lost, and in the good time I shall give them back to you. It is a hard thing that I ask, but you will do it, will you not, for Lucy's sake?" 

Arthur spoke out heartily, like his old self, "Dr. Van Helsing, you may do what you will. I feel that in saying this I am doing what my dear one would have approved. I shall not trouble you with questions till the time comes." 

The old Professor stood up as he said solemnly,"And you are right. There will be pain for us all, but it will not be all pain, nor will this pain be the last. We and you too, you most of all, dear boy, will have to pass through the bitter water before we reach the sweet. But we must be brave of heart and unselfish, and do our duty, and all will be well!" 

I slept on a sofa in Arthur's room that night. Van Helsing did not go to bed at all. He went to and fro, as if patroling the house, and was never out of sight of the room where Lucy lay in her coffin, strewn with the wild garlic flowers, which sent through the odor of lily and rose, a heavy, overpowering smell into the night. 

MINA HARKER'S JOURNAL 

22 September.--In the train to Exeter. Jonathan sleeping. It seems only yesterday that the last entry was made, and yet how much between then, in Whitby and all the world before me, Jonathan away and no news of him, and now, married to Jonathan, Jonathan a solicitor, a partner, rich, master of his business, Mr. Hawkins dead and buried, and Jonathan with another attack that may harm him. Some day he may ask me about it. Down it all goes. I am rusty in my shorthand, see what unexpected prosperity does for us, so it may be as well to freshen it up again with an exercise anyhow. 

The service was very simple and very solemn. There were only ourselves and the servants there, one or two old friends of his from Exeter, his London agent, and a gentleman representing Sir John Paxton, the President of the Incorporated Law Society. Jonathan and I stood hand in hand, and we felt that our best and dearest friend was gone from us. 

We came back to town quietly, taking a bus to Hyde Park Corner. Jonathan thought it would interest me to go into the Row for a while, so we sat down. But there were very few people there, and it was sad-looking and desolate to see so many empty chairs. It made us think of the empty chair at home. So we got up and walked down Piccadilly. Jonathan was holding me by the arm, the way he used to in the old days before I went to school. I felt it very improper, for you can't go on for some years teaching etiquette and decorum to other girls without the pedantry of it biting into yourself a bit. But it was Jonathan, and he was my husband, and we didn't know anybody who saw us, and we didn't care if they did, so on we walked. I was looking at a very beautiful girl, in a big cart-wheel hat, sitting in a victoria outside Guiliano's, when I felt Jonathan clutch my arm so tight that he hurt me, and he said under his breath, "My God!" 

I am always anxious about Jonathan, for I fear that some nervous fit may upset him again. So I turned to him quickly, and asked him what it was that disturbed him. 

He was very pale, and his eyes seemed bulging out as, half in terror and half in amazement, he gazed at a tall, thin man, with a beaky nose and black moustache and pointed beard, who was also observing the pretty girl. He was looking at her so hard that he did not see either of us, and so I had a good view of him. His face was not a good face. It was hard, and cruel, and sensual,and big white teeth, that looked all the whiter because his lips were so red, were pointed like an animal's. Jonathan kept staring at him, till I was afraid he would notice. I feared he might take it ill, he looked so fierce and nasty. I asked Jonathan why he was disturbed, and he answered, evidently thinking that I knew as much about it as he did, "Do you see who it is?" 

"No, dear," I said. "I don't know him, who is it?" His answer seemed to shock and thrill me, for it was said as if he did not know that it was me, Mina, to whom he was speaking. "It is the man himself!" 

The poor dear was evidently terrified at something, very greatly terrified. I do believe that if he had not had me to lean on and to support him he would have sunk down. He kept staring. A man came out of the shop with a small parcel, and gave it to the lady, who then drove off. Th e dark man kept his eyes fixed on her, and when the carriage moved up Piccadilly he followed in the same direction, and hailed a hansom. Jonathan kept looking after him, and said, as if to himself, 

"I believe it is the Count, but he has grown young. My God, if this be so! Oh, my God! My God! If only I knew! If only I knew!" He was distressing himself so much that I feared to keep his mind on the subject by asking him any questions, so I remained silent. I drew away quietly, and he, holding my arm, came easily. We walked a little further, and then went in and sat for a while in the Green Park. It was a hot day for autumn, and there was a comfortable seat in a shady place. After a few minutes' staring at nothing, Jonathan's eyes closed, and he went quickly into a sleep, with his head on my shoulder. I thought it was the best thing for him, so did not disturb him. In about twenty minutes he woke up, and said to me quite cheerfully, 

"Why, Mina, have I been asleep! Oh, do forgive me for being so rude. Come, and we'll have a cup of tea somewhere." 

He had evidently forgotten all about the dark stranger, as in his illness he had forgotten all that this episode had reminded him of. I don't like this lapsing into forgetfulness. It may make or continue some injury to the brain. I must not ask him, for fear I shall do more harm than good, but I must somehow learn the facts of his journey abroad. The time is come, I fear, when I must open the parcel, and know what is written. Oh, Jonathan, you will, I know, forgive me if I do wrong, but it is for your own dear sake. 

Later.--A sad home-coming in every way, the house empty of the dear soul who was so good to us. Jonathan still pale and dizzy under a slight relapse of his malady, and now a telegram from Van Helsing, whoever he may be. "You will be grieved to hear that Mrs. Westenra died five days ago, and that Lucy died the day before yesterday. They were both buried today." 

Oh, what a wealth of sorrow in a few words! Poor Mrs. Westenra! Poor Lucy! Gone, gone, never to return to us! And poor, poor Arthur, to have lost such a sweetness out of his life! God help us all to bear our troubles. 

DR. SEWARD'S DIARY-CONT. 

22 September.--It is all over. Arthur has gone back to Ring, and has taken Quincey Morris with him. What a fine fellow is Quincey! I believe in my heart of hearts that he suffered as much about Lucy's death as any of us, but he bore himself through it like a moral Viking. If America can go on breeding men like that, she will be a power in the world indeed. Van Helsing is lying down, having a rest preparatory to his journey. He goes to Amsterdam tonight, but says he returns tomorrow night, that he only wants to make some arrangements which can only be made personally. He is to stop with me then, if he can. He says he has work to do in London which may take him some time. Poor old fellow! I fear that the strain of the past week has broken down even his iron strength. All the time of the burial he was, I could see, putting some terrible restraint on himself. When it was all over, we were standing beside Arthur, who, poor fellow, was speaking of his part in the operation where his blood had been transfused to his Lucy's veins. I could see Van Helsing's face grow white and purple by turns. Arthur was saying that he felt since then as if they two had been really married, and that she was his wife in the sight of God. None of us said a word of the other operations, and none of us ever shall. Arthur and Quincey went away together to the station, and Van Helsing and I came on here. The moment we were alone in the carriage he gave way to a regular fit of hysterics. He has denied to me since that it was hysterics, and insisted that it was only his sense of humor asserting itself under very terrible conditions. He laughed till he cried, and I had to draw down the blinds lest any one should see us and misjudge. And then he cried, till he laughed again, and laughed and cried together, just as a woman does. I tried to be stern with him, as one is to a woman under the circumstances, but it had no effect. Men and women are so different in manifestations of nervous strength or weakness! Then when his face grew grave and stern again I asked him why his mirth, and why at such a time. His reply was in a way characteristic of him, for it was logical and forceful and mysterious. He said, 

"Ah, you don't comprehend, friend John. Do not think that I am not sad, though I laugh. See, I have cried even when the laugh did choke me. But no more think that I am all sorry when I cry, for the laugh he come just the same. Keep it always with you that laughter who knock at your door and say, `May I come in?' is not true laughter. No! He is a king, and he come when and how he like. He ask no person, he choose no time of suitability. He say, `I am here.' Behold, in example I grieve my heart out for that so sweet young girl. I give my blood for her, though I am old and worn. I give my time, my skill, my sleep. I let my other sufferers want that she may have all. And yet I can laugh at her very grave, laugh when the clay from the spade of the sexton drop upon her coffin and say `Thud, thud!' to my heart, till it send back the blood from my cheek. My heart bleed for that poor boy, that dear boy, so of the age of mine own boy had I been so blessed that he live, and with his hair and eyes the same. 

"There, you know now why I love him so. And yet when he say things that touch my husband-heart to the quick, and make my father-heart yearn to him as to no other man, not even you, friend John, for we are more level in experiences than father and son, yet even at such a moment King Laugh he come to me and shout and bellow in my ear,`Here I am! Here I am!' till the blood come dance back and bring some of the sunshine that he carry with him to my cheek. Oh, friend John, it is a strange world, a sad world, a world full of miseries, and woes, and troubles. And yet when King Laugh come, he make them all dance to the tune he play. Bleeding hearts, and dry bones of the churchyard, and tears that burn as they fall, all dance together to the music that he make with that smileless mouth of him. And believe me, friend John, that he is good to come, and kind. Ah, we men and women are like ropes drawn tight with strain that pull us different ways. Then tears come, and like the rain on the ropes, they brace us up, until perhaps the strain become too great, and we break. But King Laugh he come like the sunshine, and he ease off the strain again, and we bear to go on with our labor, what it may be." 

I did not like to wound him by pretending not to see his idea, but as I did not yet understand the cause of his laughter, I asked him. As he answered me his face grew stern, and he said in quite a different tone, 

"Oh, it was the grim irony of it all,this so lovely lady garlanded with flowers, that looked so fair as life, till one by one we wondered if she were truly dead, she laid in that so fine marble house in that lonely churchyard, where rest so many of her kin, laid there with the mother who loved her, and whom she loved, and that sacred bell going "Toll! Toll! Toll!' so sad and slow, and those holy men, with the white garments of the angel, pretending to read books, and yet all the time their eyes never on the page, and all of us with the bowed head. And all for what? She is dead, so! Is it not?" 

"Well, for the life of me, Professor," I said, "I can't see anything to laugh at in all that. Why, your expression makes it a harder puzzle than before. But even if the burial service was comic, what about poor Art and his trouble? Why his heart was simply breaking." 

"Just so. Said he not that the transfusion of his blood to her veins had made her truly his bride?" 

"Yes, and it was a sweet and comforting idea for him." 

"Quite so. But there was a difficulty, friend John. If so that, then what about the others? Ho, ho! Then this so sweet maid is a polyandrist, and me,with my poor wife dead to me, but alive by Church's law, though no wits, all gone, even I, who am faithful husband to this now-no-wife, am bigamist." 

"I don't see where the joke comes in there either!" I said, and I did not feel particularly pleased with him for saying such things. He laid his hand on my arm, and said, 

"Friend John, forgive me if I pain. I showed not my feeling to others when it would wound, but only to you, my old friend, whom I can trust. If you could have looked into my heart then when I want to laugh, if you could have done so when the laugh arrived, if you could do so now, when King Laugh have pack up his crown, and all that is to him, for he go far, far away from me, and for a long, long time, maybe you would perhaps pity me the most of all." 

I was touched by the tenderness of his tone, and asked why. 

"Because I know!" 

And now we are all scattered, and for many a long day loneliness will sit over our roofs with brooding wings. Lucy lies in the tomb of her kin, a lordly death house in a lonely churchyard, away from teeming London, where the air is fresh, and the sun rises over Hampstead Hill, and where wild flowers grow of their own accord. 

So I can finish this diary, and God only knows if I shall ever begin another. If I do, or if I even open this again, it will be to deal with different people and different themes, for here at the end, where the romance of my life is told, ere I go back to take up the thread of my life-work, I say sadly and without hope, "FINIS". 

THE WESTMINSTER GAZETTE, 25 SEPTEMBER A HAMPSTEAD MYSTERY 

The neighborhood of Hampstead is just at present exercised with a series of events which seem to run on lines parallel to those of what was known to the writers of headlines and "The Kensington Horror," or "The Stabbing Woman," or "The Woman in Black." During the past two or three days several cases have occurred of young children straying from home or neglecting to return from their playing on the Heath. In all these cases the children were too young to give any properly intelligible account of themselves, but the consensus of their excuses is that they had been with a "bloofer lady." It has always been late in the evening when they have been missed, and on two occasions the children have not been found until early in the following morning. It is generally supposed in the neighborhood that, as the first child missed gave as his reason for being away that a "bloofer lady" had asked him to come for a walk, the others had picked up the phrase and used it as occasion served. This is the more natural as the favorite game of the little ones at present is luring each other away by wiles. A correspondent writes us that to see some of the tiny tots pretending to be the "bloofer lady" is supremely funny. Some of our caricaturists might, he says, take a lesson in the irony of grotesque by comparing the reality and the picture. It is only in accordance with general principles of human nature that the "bloofer lady" should be the popular role at these al fresco performances. Our correspondent naively says that even Ellen Terry could not be so winningly attractive as some of these grubby-faced little children pretend, and even imagine themselves, to be. 

There is, however, possibly a serious side to the question, for some of the children, indeed all who have been missed at night, have been slightly torn or wounded in the throat. The wounds seem such as might be made by a rat or a small dog, and although of not much importance individually, would tend to show that whatever animal inflicts them has a system or method of its own. The police of the division have been instructed to keep a sharp lookout for straying children, especially when very young, in and around Hampstead Heath, and for any stray dog which may be about. 

THE WESTMINSTER GAZETTE, 25 SEPTEMBER EXTRA SPECIAL 

THE HAMPSTEAD HORROR 

ANOTHER CHILD INJURED 

THE "BLOOFER LADY" 

We have just received intelligence that another child, missed last night, was only discovered late in the morning under a furze bush at the Shooter's Hill side of Hampstead Heath, which is perhaps, less frequented than the other parts. It has the same tiny wound in the throat as has been noticed in other cases. It was terribly weak, and looked quite emaciated. It too, when partially restored, had the common story to tell of being lured away by the "bloofer lady". 
