\chapter{MINA HARKER'S JOURNAL}

23 September.--Jonathan is better after a bad night. I am so glad that he has plenty of work to do, for that keeps his mind off the terrible things, and oh, I am rejoiced that he is not now weighed down with the responsibility of his new position. I knew he would be true to himself, and now how proud I am to see my Jonathan rising to the height of his advancement and keeping pace in all ways with the duties that come upon him. He will be away all day till late, for he said he could not lunch at home. My household work is done, so I shall take his foreign journal, and lock myself up in my room and read it. 

24 September.--I hadn't the heart to write last night, that terrible record of Jonathan's upset me so. Poor dear! How he must have suffered, whether it be true or only imagination. I wonder if there is any truth in it at all. Did he get his brain fever, and then write all those terrible things, or had he some cause for it all? I suppose I shall never know, for I dare not open the subject to him. And yet that man we saw yesterday! He seemed quite certain of him, poor fellow! I suppose it was the funeral upset him and sent his mind back on some train of thought. 

He believes it all himself. I remember how on our wedding day he said "Unless some solemn duty come upon me to go back to the bitter hours, asleep or awake, mad or sane . . ." There seems to be through it all some thread of continuity. That fearful Count was coming to London. If it should be, and he came to London, with its teeming millions . . . There may be a solemn duty, and if it come we must not shrink from it. I shall be prepared. I shall get my typewriter this very hour and begin transcribing. Then we shall be ready for other eyes if required. And if it be wanted, then, perhaps, if I am ready, poor Jonathan may not be upset, for I can speak for him and never let him be troubled or worried with it at all. If ever Jonathan quite gets over the nervousness he may want to tell me of it all, and I can ask him questions and find out things, and see how I may comfort him. 

LETTER, VAN HELSING TO MRS. HARKER 

24 September 

(Confidence) 

"Dear Madam, 

"I pray you to pardon my writing, in that I am so far friend as that I sent to you sad news of Miss Lucy Westenra's death. By the kindness of Lord Godalming, I am empowered to read her letters and papers, for I am deeply concerned about certain matters vitally important. In them I find some letters from you, which show how great friends you were and how you love her. Oh, Madam Mina, by that love, I implore you, help me. It is for others' good that I ask, to redress great wrong, and to lift much and terrible troubles, that may be more great than you can know. May it be that I see you? You can trust me. I am friend of Dr. John Seward and of Lord Godalming (that was Arthur of Miss Lucy). I must keep it private for the present from all. I should come to Exeter to see you at once if you tell me I am privilege to come, and where and when. I implore your pardon, Madam. I have read your letters to poor Lucy, and know how good you are and how your husband suffer. So I pray you, if it may be, enlighten him not, least it may harm. Again your pardon, and forgive me. 

"VAN HELSING" 

TELEGRAM, MRS. HARKER TO VAN HELSING 

25 September.--Come today by quarter past ten train if you can catch it. Can see you any time you call. "WILHELMINA HARKER" 

MINA HARKER'S JOURNAL 

25 September.--I cannot help feeling terribly excited as the time draws near for the visit of Dr. Van Helsing, for somehow I expect that it will throw some light upon Jonathan's sad experience, and as he attended poor dear Lucy in her last illness, he can tell me all about her. That is the reason of his coming. It is concerning Lucy and her sleepwalking, and not about Jonathan. Then I shall never know the real truth now! How silly I am. That awful journal gets hold of my imagination and tinges everything with something of its own color. Of course it is about Lucy. That habit came back to the poor dear, and that awful night on the cliff must have made her ill. I had almost forgotten in my own affairs how ill she was afterwards. She must have told him of her sleep-walking adventure on the cliff, and that I knew all about it, and now he wants me to tell him what I know, so that he may understand. I hope I did right in not saying anything of it to Mrs. Westenra. I should never forgive myself if any act of mine, were it even a negative one, brought harm on poor dear Lucy. I hope too, Dr. Van Helsing will not blame me. I have had so much trouble and anxiety of late that I feel I cannot bear more just at present. 

I suppose a cry does us all good at times, clears the air as other rain does. Perhaps it was reading the journal yesterday that upset me, and then Jonathan went away this morning to stay away from me a whole day and night, the first time we have been parted since our marriage. I do hope the dear fellow will take care of himself, and that nothing will occur to upset him. It is two o'clock, and the doctor will be here soon now. I shall say nothing of Jonathan's journal unless he asks me. I am so glad I have typewritten out my own journal, so that, in case he asks about Lucy, I can hand it to him. It will save much questioning. 

Later.--He has come and gone. Oh, what a strange meeting, and how it all makes my head whirl round. I feel like one in a dream. Can it be all possible, or even a part of it? If I had not read Jonathan's journal first, I should never have accepted even a possibility. Poor, poor, dear Jonathan! How he must have suffered. Please the good God, all this may not upset him again. I shall try to save him from it. But it may be even a consolation and a help to him, terrible though it be and awful in its consequences, to know for certain that his eyes and ears and brain did not deceive him, and that it is all true. It may be that it is the doubt which haunts him, that when the doubt is removed, no matter which, waking or dreaming, may prove the truth, he will be more satisfied and better able to bear the shock. Dr. Van Helsing must be a good man as well as a clever one if he is Arthur's friend and Dr. Seward's, and if they brought him all the way from Holland to look after Lucy. I feel from having seen him that he is good and kind and of a noble nature. When he comes tomorrow I shall ask him about Jonathan. And then, please God, all this sorrow and anxiety may lead to a good end. I used to think I would like to practice interviewing. Jonathan's friend on "The Exeter News" told him that memory is everything in such work, that you must be able to put down exactly almost every word spoken, even if you had to refine some of it afterwards. Here was a rare interview. I shall try to record it verbatim. 

It was half-past two o'clock when the knock came. I took my courage a deux mains and waited. In a few minutes Mary opened the door, and announced "Dr. Van Helsing". 

I rose and bowed, and he came towards me, a man of medium weight, strongly built, with his shoulders set back over a broad, deep chest and a neck well balanced on the trunk as the head is on the neck. The poise of the head strikes me at once as indicative of thought and power. The head is noble, well-sized, broad, and large behind the ears. The face, cleanshaven, shows a hard, square chin, a large resolute, mobile mouth, a good-sized nose, rather straight, but with quick, sensitive nostrils, that seem to broaden as the big bushy brows come down and the mouth tightens. The forehead is broad and fine, rising at first almost straight and then sloping back above two bumps or ridges wide apart, such a forehead that the reddish hair cannot possibly tumble over it, but falls naturally back and to the sides. Big, dark blue eyes are set widely apart, and are quick and tender or stern with the man's moods. He said to me, 

"Mrs. Harker, is it not?" I bowed assent. 

"That was Miss Mina Murray?" Again I assented. 

"It is Mina Murray that I came to see that was friend of that poor dear child Lucy Westenra. Madam Mina, it is on account of the dead that I come." 

"Sir," I said, "you could have no better claim on me than that you were a friend and helper of Lucy Westenra."And I held out my hand. He took it and said tenderly, 

"Oh, Madam Mina, I know that the friend of that poor little girl must be good, but I had yet to learn . . ." He finished his speech with a courtly bow. I asked him what it was that he wanted to see me about, so he at once began. 

"I have read your letters to Miss Lucy. Forgive me, but I had to begin to inquire somewhere, and there was none to ask. I know that you were with her at Whitby. She sometimes kept a diary, you need not look surprised, Madam Mina. It was begun after you had left, and was an imitation of you, and in that diary she traces by inference certain things to a sleep-walking in which she puts down that you saved her. In great perplexity then I come to you, and ask you out of your so much kindness to tell me all of it that you can remember." 

"I can tell you, I think, Dr. Van Helsing, all about it." 

"Ah, then you have good memory for facts, for details? It is not always so with young ladies." 

"No, doctor, but I wrote it all down at the time. I can show it to you if you like." 

"Oh, Madam Mina, I well be grateful. You will do me much favor." 

I could not resist the temptation of mystifying him a bit, I suppose it is some taste of the original apple that remains still in our mouths, so I handed him the shorthand diary. He took it with a grateful bow, and said, "May I read it?" 

"If you wish," I answered as demurely as I could. He opened it, and for an instant his face fell. Then he stood up and bowed. 

"Oh, you so clever woman!" he said. "I knew long that Mr. Jonathan was a man of much thankfulness, but see, his wife have all the good things. And will you not so much honor me and so help me as to read it for me? Alas! I know not the shorthand." 

By this time my little joke was over, and I was almost ashamed. So I took the typewritten copy from my work basket and handed it to him. 

"Forgive me," I said. "I could not help it, but I had been thinking that it was of dear Lucy that you wished to ask, and so that you might not have time to wait, not on my account, but because I know your time must be precious, I have written it out on the typewriter for you." 

He took it and his eyes glistened. "You are so good," he said. "And may I read it now? I may want to ask you some things when I have read." 

"By all means," I said. "read it over whilst I order lunch, and then you can ask me questions whilst we eat." 

He bowed and settled himself in a chair with his back to the light, and became so absorbed in the papers, whilst I went to see after lunch chiefly in order that he might not be disturbed. When I came back, I found him walking hurriedly up and down the room, his face all ablaze with excitement. He rushed up to me and took me by both hands. 

"Oh, Madam Mina," he said, "how can I say what I owe to you? This paper is as sunshine. It opens the gate to me. I am dazed, I am dazzled, with so much light, and yet clouds roll in behind the light every time. But that you do not, cannot comprehend. Oh, but I am grateful to you, you so clever woman. Madame," he said this very solemnly, "if ever Abraham Van Helsing can do anything for you or yours, I trust you will let me know. It will be pleasure and delight if I may serve you as a friend, as a friend, but all I have ever learned, all I can ever do, shall be for you and those you love. There are darknesses in life, and there are lights. You are one of the lights. You will have a happy life and a good life, and your husband will be blessed in you." 

"But, doctor, you praise me too much, and you do not know me." 

"Not know you, I, who am old, and who have studied all my life men and women, I who have made my specialty the brain and all that belongs to him and all that follow from him! And I have read your diary that you have so goodly written for me, and which breathes out truth in every line. I, who have read your so sweet letter to poor Lucy of your marriage and your trust, not know you! Oh, Madam Mina, good women tell all their lives, and by day and by hour and by minute, such things that angels can read. And we men who wish to know have in us something of angels' eyes. Your husband is noble nature, and you are noble too, for you trust, and trust cannot be where there is mean nature. And your husband, tell me of him. Is he quite well? Is all that fever gone, and is he strong and hearty?" 

I saw here an opening to ask him about Jonathan, so I said,"He was almost recovered, but he has been greatly upset by Mr. Hawkins death." 

He interrupted, "Oh, yes. I know. I know. I have read your last two letters." 

I went on, "I suppose this upset him, for when we were in town on Thursday last he had a sort of shock." 

"A shock, and after brain fever so soon! That is not good. What kind of shock was it?" 

"He thought he saw some one who recalled something terrible, something which led to his brain fever." And here the whole thing seemed to overwhelm me in a rush. The pity for Jonathan, the horror which he experienced, the whole fearful mystery of his diary, and the fear that has been brooding over me ever since, all came in a tumult. I suppose I was hysterical, for I threw myself on my knees and held up my hands to him, and implored him to make my husband well again. He took my hands and raised me up, and made me sit on the sofa, and sat by me. He held my hand in his, and said to me with, oh, such infinite sweetness, 

"My life is a barren and lonely one, and so full of work that I have not had much time for friendships, but since I have been summoned to here by my friend John Seward I have known so many good people and seen such nobility that I feel more than ever, and it has grown with my advancing years, the loneliness of my life. Believe me, then, that I come here full of respect for you, and you have given me hope, hope, not in what I am seeking of, but that there are good women still left to make life happy, good women, whose lives and whose truths may make good lesson for the children that are to be. I am glad, glad, that I may here be of some use to you. For if your husband suffer, he suffer within the range of my study and experience. I promise you that I will gladly do all for him that I can, all to make his life strong and manly, and your life a happy one. Now you must eat. You are over-wrought and perhaps over-anxious. Husband Jonathan would not like to see you so pale, and what he like not where he love, is not to his good. Therefore for his sake you must eat and smile. You have told me about Lucy, and so now we shall not speak of it, lest it distress. I shall stay in Exeter tonight, for I want to think much over what you have told me, and when I have thought I will ask you questions, if I may. And then too, you will tell me of husband Jonathan's trouble so far as you can, but not yet. You must eat now, afterwards you shall tell me all." 

After lunch, when we went back to the drawing room, he said to me, "And now tell me all about him." 

When it came to speaking to this great learned man, I began to fear that he would think me a weak fool, and Jonathan a madman, that journal is all so strange, and I hesitated to go on. But he was so sweet and kind, and he had promised to help, and I trusted him, so I said, 

"Dr. Van Helsing, what I have to tell you is so queer that you must not laugh at me or at my husband. I have been since yesterday in a sort of fever of doubt. You must be kind to me, and not think me foolish that I have even half believed some very strange things." 

He reassured me by his manner as well as his words when he said, "Oh, my dear, if you only know how strange is the matter regarding which I am here, it is you who would laugh. I have learned not to think little of any one's belief, no matter how strange it may be. I have tried to keep an open mind, and it is not the ordinary things of life that could close it, but the strange things, the extraordinary things, the things that make one doubt if they be mad or sane." 

"Thank you, thank you a thousand times! You have taken a weight off my mind. If you will let me, I shall give you a paper to read. It is long, but I have typewritten it out. It will tell you my trouble and Jonathan's. It is the copy of his journal when abroad, and all that happened. I dare not say anything of it. You will read for yourself and judge. And then when I see you, perhaps, you will be very kind and tell me what you think." 

"I promise," he said as I gave him the papers. "I shall in the morning, as soon as I can, come to see you and your husband, if I may." 

"Jonathan will be here at half-past eleven, and you must come to lunch with us and see him then. You could catch the quick 3:34 train, which will leave you at Paddington before eight." He was surprised at my knowledge of the trains offhand, but he does not know that I have made up all the trains to and from Exeter, so that I may help Jonathan in case he is in a hurry. 

So he took the papers with him and went away, and I sit here thinking, thinking I don't know what. 

LETTER (by hand), VAN HELSING TO MRS. HARKER 

25 September, 6 o'clock 

"Dear Madam Mina, 

"I have read your husband's so wonderful diary. You may sleep without doubt. Strange and terrible as it is, it is true! I will pledge my life on it. It may be worse for others, but for him and you there is no dread. He is a noble fellow, and let me tell you from experience of men, that one who would do as he did in going down that wall and to that room, aye, and going a second time, is not one to be injured in permanence by a shock. His brain and his heart are all right, this I swear, before I have even seen him, so be at rest. I shall have much to ask him of other things. I am blessed that today I come to see you, for I have learn all at once so much that again I am dazzled, dazzled more than ever, and I must think. 

"Yours the most faithful, 

"Abraham Van Helsing." 

LETTER, MRS. HARKER TO VAN HELSING 

25 September, 6:30 p. m. 

"My dear Dr. Van Helsing, 

"A thousand thanks for your kind letter, which has taken a great weight off my mind. And yet, if it be true, what terrible things there are in the world, and what an awful thing if that man, that monster, be really in London! I fear to think. I have this moment, whilst writing, had a wire from Jonathan, saying that he leaves by the 6:25 tonight from Launceston and will be here at 10:18,so that I shall have no fear tonight. Will you, therefore, instead of lunching with us, please come to breakfast at eight o'clock, if this be not too early for you? You can get away, if you are in a hurry, by the 10:30 train, which will bring you to Paddington by 2:35. Do not answer this, as I shall take it that, if I do not hear, you will come to breakfast. 

"Believe me, 

"Your faithful and grateful friend, 

"Mina Harker." 

JONATHAN HARKER'S JOURNAL 

26 September.--I thought never to write in this diary again, but the time has come. When I got home last night Mina had supper ready, and when we had supped she told me of Van Helsing's visit, and of her having given him the two diaries copied out, and of how anxious she has been about me. She showed me in the doctor's letter that all I wrote down was true. It seems to have made a new man of me. It was the doubt as to the reality of the whole thing that knocked me over. I felt impotent, and in the dark, and distrustful. But, now that I know, I am not afraid, even of the Count. He has succeeded after all, then, in his design in getting to London, and it was he I saw. He has got younger, and how? Van Helsing is the man to unmask him and hunt him out, if he is anything like what Mina says. We sat late, and talked it over. Mina is dressing, and I shall call at the hotel in a few minutes and bring him over. 

He was, I think, surprised to see me. When I came into the room whee he was, and introduced myself, he took me by the shoulder, and turned my face round to the light, and said, after a sharp scrutiny, 

"But Madam Mina told me you were ill, that you had had a shock." 

It was so funny to hear my wife called `Madam Mina' by this kindly, strong-faced old man. I smiled, and said, "I was ill, I have had a shock, but you have cured me already." 

"And how?" 

"By your letter to Mina last night. I was in doubt, and then everything took a hue of unreality, and I did not know what to trust, even the evidence of my own senses. Not knowing what to trust, I did not know what to do, and so had only to keep on working in what had hitherto been the groove of my life. The groove ceased to avail me, and I mistrusted myself. Doctor, you don't know what it is to doubt everything, even yourself. No, you don't, you couldn't with eyebrows like yours." 

He seemed pleased, and laughed as he said, "So! You are a physiognomist. I learn more here with each hour. I am with so much pleasure coming to you to breakfast, and, oh, sir, you will pardon praise from an old man, but you are blessed in your wife." 

I would listen to him go on praising Mina for a day, so I simply nodded and stood silent. 

"She is one of God's women, fashioned by His own hand to show us men and other women that there is a heaven where we can enter, and that its light can be here on earth. So true, so sweet, so noble, so little an egoist, and that, let me tell you, is much in this age, so sceptical and selfish. And you, sir. . . I have read all the letters to poor Miss Lucy, and some of them speak of you, so I know you since some days from the knowing of others, but I have seen your true self since last night. You will give me your hand, will you not? And let us be friends for all our lives." 

We shook hands, and he was so earnest and so kind that it made me quite choky. 

"and now," he said, "may I ask you for some more help? I have a great task to do, and at the beginning it is to know. You can help me here. Can you tell me what went before your going to Transylvania? Later on I may ask more help, and of a different kind, but at first this will do." 

"Look here, Sir," I said, "does what you have to do concern the Count?" 

"It does," he said solemnly." 

"Then I am with you heart and soul. As you go by the 10:30 train, you will not have time to read them, but I shall get the bundle of papers. You can take them with you and read them in the train." 

After breakfast I saw him to the station. When we were parting he said, "Perhaps you will come to town if I send for you, and take Madam Mina too." 

"We shall both come when you will," I said. 

I had got him the morning papers and the London papers of the previous night, and while we were talking at the carriage window, waiting for the train to start, he was turning them over. His eyes suddenly seemed to catch something in one of them, "The Westminster Gazette", I knew it by the color, and he grew quite white. He read something intently, groaning to himself, "Mein Gott! Mein Gott! So soon! So soon!" I do not think he remembered me at the moment. Just then the whistle blew, and the train moved off. This recalled him to himself, and he leaned out of the window and waved his hand, calling out, "Love to Madam Mina. I shall write so soon as ever I can." 

DR. SEWARD'S DIARY 

26 September.--Truly there is no such thing as finality. Not a week since I said "Finis," and yet here I am starting fresh again, or rather going on with the record. Until this afternoon I had no cause to think of what is done. Renfield had become, to all intents, as sane as he ever was. He was already well ahead with his fly business, and he had just started in the spider line also, so he had not been of any trouble to me. I had a letter from Arthur, written on Sunday, and from it I gather that he is bearing up wonderfully well. Quincey Morris is with him, and that is much of a help, for he himself is a bubbling well of good spirits. Quincey wrote me a line too, and from him I hear that Arthur is beginning to recover something of his old buoyancy, so as to them all my mind is at rest. As for myself, I was settling down to my work with the enthusiasm which I used to have for it, so that I might fairly have said that the wound which poor Lucy left on me was becoming cicatrised. 

Everything is, however, now reopened, and what is to be the end God only knows. I have an idea that Van Helsing thinks he knows, too, but he will only let out enough at a time to whet curiosity. He went to Exeter yesterday, and stayed there all night. Today he came back, and almost bounded into the room at about half-past five o'clock, and thrust last night's "Westminster Gazette" into my hand. 

"What do you think of that?" he asked as he stood back and folded his arms. 

I looked over the paper, for I really did not know what he meant, but he took it from me and pointed out a paragraph about children being decoyed away at Hampstead. It did not convey much to me, until I reached a passage where it described small puncture wounds on their throats. An idea struck me, and I looked up. 

"Well?" he said. 

"It is like poor Lucy's." 

"And what do you make of it?" 

"Simply that there is some cause in common. Whatever it was that injured her has injured them." I did not quite understand his answer. 

"That is true indirectly, but not directly." 

"How do you mean, Professor?" I asked. I was a little inclined to take his seriousness lightly, for, after all, four days of rest and freedom from burning, harrowing, anxiety does help to restore one's spirits, but when I saw his face, it sobered me. Never, even in the midst of our despair about poor Lucy, had he looked more stern. 

"Tell me!" I said. "I can hazard no opinion. I do not know what to think, and I have no data on which to found a conjecture." 

"Do you mean to tell me, friend John, that you have no suspicion as to what poor Lucy died of, not after all the hints given, not only by events, but by me?" 

"Of nervous prostration following a great loss or waste of blood." 

"And how was the blood lost or wasted?" I shook my head. 

He stepped over and sat down beside me, and went on,"You are a clever man, friend John. You reason well, and your wit is bold, but you are too prejudiced. You do not let your eyes see nor your ears hear, and that which is outside your daily life is not of account to you. Do you not think that there are things which you cannot understand, and yet which are,that some people see things that others cannot? But there are things old and new which must not be contemplated by men's eyes, because they know, or think they know, some things which other men have told them. Ah, it is the fault of our science that it wants to explain all, and if it explain not, then it says there is nothing to explain. But yet we see around us every day the growth of new beliefs, which think themselves new, and which are yet but the old, which pretend to be young, like the fine ladies at the opera. I suppose now you do not believe in corporeal transference. No? Nor in materialization. No? Nor in astral bodies. No? Nor in the reading of thought. No? Nor in hypnotism . . ." 

"Yes," I said. "Charcot has proved that pretty well." 

He smiled as he went on, "Then you are satisfied as to it. Yes? And of course then you understand how it act, and can follow the mind of the great Charcot, alas that he is no more, into the very soul of the patient that he influence. No? Then, friend John, am I to take it that you simply accept fact, and are satisfied to let from premise to conclusion be a blank? No? Then tell me, for I am a student of the brain, how you accept hypnotism and reject the thought reading. Let me tell you, my friend, that there are things done today in electrical science which would have been deemed unholy by the very man who discovered electricity, who would themselves not so long before been burned as wizards. There are always mysteries in life. Why was it that Methuselah lived nine hundred years, and `Old Parr'one hundred and sixty-nine, and yet that poor Lucy, with four men's blood in her poor veins, could not live even one day? For, had she live one more day, we could save her. Do you know all the mystery of life and death? Do you know the altogether of comparative anatomy and can say wherefore the qualities of brutes are in some men, and not in others? Can you tell me why, when other spiders die small and soon, that one great spider lived for centuries in the tower of the old Spanish church and grew and grew, till, on descending, he could drink the oil of all the church lamps? Can you tell me why in the Pampas, ay and elsewhere, there are bats that come out at night and open the veins of cattle and horses and suck dry their veins, how in some islands of the Western seas there are bats which hang on the trees all day, and those who have seen describe as like giant nuts or pods, and that when the sailors sleep on the deck, because that it is hot, flit down on them and then, and then in the morning are found dead men, white as even Miss Lucy was?" 

"Good God, Professor!" I said, starting up. "Do you mean to tell me that Lucy was bitten by such a bat, and that such a thing is here in London in the nineteenth century?" 

He waved his hand for silence, and went on,"Can you tell me why the tortoise lives more long than generations of men, why the elephant goes on and on till he have sees dynasties, and why the parrot never die only of bite of cat of dog or other complaint? Can you tell me why men believe in all ages and places that there are men and women who cannot die? We all know, because science has vouched for the fact, that there have been toads shut up in rocks for thousands of years, shut in one so small hole that only hold him since the youth of the world. Can you tell me how the Indian fakir can make himself to die and have been buried, and his grave sealed and corn sowed on it, and the corn reaped and be cut and sown and reaped and cut again, and then men come and take away the unbroken seal and that there lie the Indian fakir, not dead, but that rise up and walk amongst them as before?" 

Here I interrupted him. I was getting bewildered. He so crowded on my mind his list of nature's eccentricities and possible impossibilities that my imagination was getting fired. I had a dim idea that he was teaching me some lesson, as long ago he used to do in his study at Amsterdam. But he used them to tell me the thing, so that I could have the object of thought in mind all the time. But now I was without his help, yet I wanted to follow him, so I said, 

"Professor, let me be your pet student again. Tell me the thesis, so that I may apply your knowledge as you go on. At present I am going in my mind from point to point as a madman, and not a sane one, follows an idea. I feel like a novice lumbering through a bog in a midst, jumping from one tussock to another in the mere blind effort to move on without knowing where I am going." 

"That is a good image," he said. "Well, I shall tell you. My thesis is this, I want you to believe." 

"To believe what?" 

"To believe in things that you cannot. Let me illustrate. I heard once of an American who so defined faith, `that fac ulty which enables us to believe things which we know to be untrue.' For one, I follow that man. He meant that we shall have an open mind, and not let a little bit of truth check the rush of the big truth, like a small rock does a railway truck. We get the small truth first. Good! We keep him, and we value him, but all the same we must not let him think himself all the truth in the universe." 

"Then you want me not to let some previous conviction inure the receptivity of my mind with regard to some strange matter. Do I read your lesson aright?" 

"Ah, you are my favorite pupil still. It is worth to teach you. Now that you are willing to understand, you have taken the first step to understand. You think then that those so small holes in the children's throats were made by the same that made the holes in Miss Lucy?" 

"I suppose so." 

He stood up and said solemnly, "Then you are wrong. Oh, would it were so! But alas! No. It is worse, far, far worse." 

"In God's name, Professor Van Helsing, what do you mean?" I cried. 

He threw himself with a despairing gesture into a chair, and placed his elbows on the table, covering his face with his hands as he spoke. 

"They were made by Miss Lucy!" 
