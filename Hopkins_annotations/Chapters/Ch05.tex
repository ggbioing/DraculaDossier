LETTER FROM MISS MINA MURRAY TO MISS LUCY WESTENRA 

9 May. 

My dearest Lucy, 

Forgive my long delay in writing, but I have been simply overwhelmed with work. The life of an assistant schoolmistress is sometimes trying. I am longing to be with you, and by the sea, where we can talk together freely and build our castles in the air. I have been working very hard lately, because I want to keep up with Jonathan's studies, and I have been practicing shorthand very assiduously. When we are married I shall be able to be useful to Jonathan, and if I can stenograph well enough I can take down what he wants to say in this way and write it out for him on the typewriter, at which also I am practicing very hard. 

He and I sometimes write letters in shorthand, and he is keeping a stenographic journal of his travels abroad. When I am with you I shall keep a diary in the same way. I don't mean one of those two-pages-to-the-week-with-Sunday-squeezed-in-a-corner diaries, but a sort of journal which I can write in whenever I feel inclined. 

I do not suppose there will be much of interest to other people, but it is not intended for them. I may show it to Jonathan some day if there is in it anything worth sharing, but it is really an exercise book. I shall try to do what I see lady journalists do, interviewing and writing descriptions and trying to remember conversations. I am told that, with a little practice, one can remember all that goes on or that one hears said during a day. 

However, we shall see. I will tell you of my little plans when we meet. I have just had a few hurried lines from Jonathan from Transylvania. He is well, and will be returning in about a week. I am longing to hear all his news. It must be nice to see strange countries. I wonder if we, I mean Jonathan and I, shall ever see them together. There is the ten o'clock bell ringing. Goodbye. Your loving Mina 

Tell me all the news when you write. You have not told me anything for a long time. I hear rumours, and especially of a tall, handsome, curly-haired man.??? 

LETTER, LUCY WESTENRA TO MINA MURRAY 

17, Chatham Street 

Wednesday 

My dearest Mina, 

I must say you tax me very unfairly with being a bad correspondent. I wrote you twice since we parted, and your last letter was only your second. Besides, I have nothing to tell you. There is really nothing to interest you. 

Town is very pleasant just now, and we go a great deal to picture-galleries and for walks and rides in the park. As to the tall, curly-haired man, I suppose it was the one who was with me at the last Pop. Someone has evidently been telling tales. 

That was Mr. Holmwood. He often comes to see us, and he and Mamma get on very well together, they have so many things to talk about in common. 

We met some time ago a man that would just do for you, if you were not already engaged to Jonathan. He is an excellant parti, being handsome, well off, and of good birth. He is a doctor and really clever. Just fancy! He is only nine-and twenty, and he has an immense lunatic asylum all under his own care. Mr. Holmwood introduced him to me, and he called here to see us, and often comes now. I think he is one of the most resolute men I ever saw, and yet the most calm. He seems absolutely imperturbable. I can fancy what a wonderful power he must have over his patients. He has a curious habit of looking one straight in the face, as if trying to read one's thoughts. He tries this on very much with me, but I flatter myself he has got a tough nut to crack. I know that from my glass. 

Do you ever try to read your own face? I do, and I can tell you it is not a bad study, and gives you more trouble than you can well fancy if you have never tried it. 

He say that I afford him a curious psychological study, and I humbly think I do. I do not, as you know, take sufficient interest in dress to be able to describe the new fashions. Dress is a bore. That is slang again, but never mind. Arthur says that every day. 

There, it is all out, Mina, we have told all our secrets to each other since we were children. We have slept together and eaten together, and laughed and cried together, and now, though I have spoken, I would like to speak more. Oh, Mina, couldn't you guess? I love him. I am blushing as I write, for although I think he loves me, he has not told me so in words. But, oh, Mina, I love him. I love him! There, that does me good. 

I wish I were with you, dear, sitting by the fire undressing, as we used to sit, and I would try to tell you what I feel. I do not know how I am writing this even to you. I am afraid to stop, or I should tear up the letter, and I don't want to stop, for I do so want to tell you all. Let me hear from you at once, and tell me all that you think about it. Mina, pray for my happiness. 

Lucy 

P. S.--I need not tell you this is a secret. Goodnight again. L. 

LETTER, LUCY WESTENRA TO MINA MURRAY 

24 May 

My dearest Mina, 

Thanks, and thanks, and thanks again for your sweet letter. It was so nice to be able to tell you and to have your sympathy. My dear, it never rains but it pours. How true the old proverbs are. Here am I, who shall be twenty in September, and yet I never had a proposal till today, not a real proposal, and today I had three. Just fancy! Three proposals in one day! Isn't it awful! I feel sorry, really and truly sorry, for two of the poor fellows. Oh, Mina, I am so happy that I don't know what to do with myself. And three proposals! But, for goodness' sake, don't tell any of the girls, or they would be getting all sorts of extravagant ideas, and imagining themselves injured and slighted if in their very first day at home they did not get six at least. Some girls are so vain! You and I, Mina dear, who are engaged and are going to settle down soon soberly into old married women, can despise vanity. Well, I must tell you about the three, but you must keep it a secret, dear, from every one except, of course, Jonathan. You will tell him, because I would, if I were in your place, certainly tell Arthur. A woman ought to tell her husband everything. Don't you think so, dear? And I must be fair. Men like women, certainly their wives, to be quite as fair as they are. And women, I am afraid, are not always quite as fair as they should be. 

Well, my dear, number One came just before lunch. I told you of him, Dr. John Seward, the lunatic asylum man, with the strong jaw and the good forehead. He was very cool outwardly, but was nervous all the same. He had evidently been schooling himself as to all sorts of little things, and remembered them, but he almost managed to sit down on his silk hat, which men don't generally do when they are cool, and then when he wanted to appear at ease he kept playing with a lancet in a way that made me nearly scream. He spoke to me, Mina, very straightfordwardly. He told me how dear I was to him, though he had known me so little, and what his life would be with me to help and cheer him. He was going to tell me how unhappy he would be if I did not care for him, but when he saw me cry he said he was a brute and would not add to my present trouble. Then he broke off and asked if I could love him in time, and when I shook my head his hands trembled, and then with some hesitation he asked me if I cared already for any one else. He put it very nicely, saying that he did not want to wring my confidence from me, but only to know, because if a woman's heart was free a man might have hope. And then, Mina, I felt a sort of duty to tell him that there was some one. I only told him that much, and then he stood up, and he looked very strong and very grave as he took both my hands in his and said he hoped I would be happy, and that If I ever wanted a friend I must count him one of my best. 

Oh, Mina dear, I can't help crying, and you must excuse this letter being all blotted. Being proposed to is all very nice and all that sort of thing, but it isn't at all a happy thing when you have to see a poor fellow, whom you know loves you honestly, going away and looking all broken hearted, and to know that, no matter what he may say at the moment, you are passing out of his life. My dear, I must stop here at present, I feel so miserable, though I am so happy. 

Evening. 

Arthur has just gone, and I feel in better spirits than when I left off, so I can go on telling you about the day. 

Well, my dear, number Two came after lunch. He is such a nice fellow, and American from Texas, and he looks so young and so fresh that it seems almost impossible that he has been to so many places and has such adventures. I sympathize with poor Desdemona when she had such a stream poured in her ear, even by a black man. I suppose that we women are such cowards that we think a man will save us from fears, and we marry him. I know now what I would do if I were a man and wanted to make a girl love me. No, I don't, for there was Mr. Morris telling us his stories, and Arthur never told any, and yet . . . 

My dear, I am somewhat previous. Mr. Quincy P. Morris found me alone. It seems that a man always does find a girl alone. No, he doesn't, for Arthur tried twice to make a chance, and I helping him all I could, I am not ashamed to say it now. I must tell you beforehand that Mr. Morris doesn't always speak slang, that is to say, he never does so to strangers or before them, for he is really well educated and has exquisite manners, but he found out that it amused me to hear him talk American slang, and whenever I was present, and there was no one to be shocked, he said such funny things. I am afraid, my dear, he has to invent it all, for it fits exactly into whatever else he has to say. But this is a way slang has. I do not know myself if I shall ever speak slang. I do not know if Arthur likes it, as I have never heard him use any as yet. 

Well, Mr. Morris sat down beside me and looked as happy and jolly as he could, but I could see all the same that he was very nervous. He took my hand in his, and said ever so sweetly . . . 

"Miss Lucy, I know I ain't good enough to regulate the fixin's of your little shoes, but I guess if you wait till you find a man that is you will go join them seven young women with the lamps when you quit. Won't you just hitch up along-side of me and let us go down the long road together, driving in double harness?" 

Well, he did look so hood humoured and so jolly that it didn't seem half so hard to refuse him as it did poor Dr. Seward. So I said, as lightly as I could, that I did not know anything of hitching, and that I wasn't broken to harness at all yet. Then he said that he had spoken in a light manner, and he hoped that if he had made a mistake in doing so on so grave, so momentous, and occasion for him, I would forgive him. He really did look serious when he was saying it, and I couldn't help feeling a sort of exultation that he was number Two in one day. And then, my dear, before I could say a word he began pouring out a perfect torrent of lovemaking, laying his very heart and soul at my feet. He looked so earnest over it that I shall never again think that a man must be playful always, and never earnest, because he is merry at times. I suppose he saw something in my face which checked him, for he suddenly stopped, and said with a sort of manly fervour that I could have loved him for if I had been free . . . 

"Lucy, you are an honest hearted girl, I know. I should not be here speaking to you as I am now if I did not believe you clean grit, right through to the very depths of your soul. Tell me, like one good fellow to another, is there any one else that you care for? And if there is I'll never trouble you a hair's breadth again, but will be, if you will let me, a very faithful friend." 

My dear Mina, why are men so noble when we women are so little worthy of them? Here was I almost making fun of this great hearted, true gentleman. I burst into tears, I am afraid, my dear, you will think this a very sloppy letter in more ways than one, and I really felt very badly. 

Why can't they let a girl marry three men, or as many as want her, and save all this trouble? But this is heresy, and I must not say it. I am glad to say that, though I was crying, I was able to look into Mr. Morris' brave eyes, and I told him out straight . . . 

"Yes, there is some one I love, though he has not told me yet that he even loves me." I was right to speak to him so frankly, for quite a light came into his face, and he put out both his hands and took mine, I think I put them into his, and said in a hearty way . . . 

"That's my brave girl. It's better worth being late for a chance of winning you than being in time for any other girl in the world. Don't cry, my dear. If it's for me, I'm a hard nut to crack, and I take it standing up. If that other fellow doesn't know his happiness, well, he'd better look for it soon, or he'll have to deal with me. Little girl, your honesty and pluck have made me a friend, and that's rarer than a lover, it's more selfish anyhow. My dear, I'm going to have a pretty lonely walk between this and Kingdom Come. Won't you give me one kiss? It'll be something to keep off the darkness now and then. You can, you know, if you like, for that other good fellow, or you could not love him, hasn't spoken yet." 

That quite won me, Mina, for it was brave and sweet of him, and noble too, to a rival, wasn't it? And he so sad, so I leant over and kissed him. 

He stood up with my two hands in his, and as he looked down into my face, I am afraid I was blushing very much, he said, "Little girl, I hold your hand, and you've kissed me, and if these things don't make us friends nothing ever will. Thank you for your sweet honesty to me, and goodbye." He wrung my hand, and taking up his hat, went straight out of the room without looking back, without a tear or a quiver or a pause, and I am crying like a baby. 

Oh, why must a man like that be made unhappy when there are lots of girls about who would worship the very ground he trod on? I know I would if I were free, only I don't want to be free My dear, this quite upset me, and I feel I cannot write of happiness just at once, after telling you of it, and I don't wish to tell of the number Three until it can be all happy. Ever your loving . . . Lucy 

P. S.--Oh, about number Three, I needn't tell you of number Three, need I? Besides, it was all so confused. It seemed only a moment from his coming into the room till both his arms were round me, and he was kissing me. I am very, very happy, and I don't know what I have done to deserve it. I must only try in the future to show that I am not ungrateful to God for all His goodness to me in sending to me such a lover, such a husband, and such a friend. 

Goodbye. 

DR. SEWARD'S DIARY (Kept in phonograph) 

25 May.--Ebb tide in appetite today. Cannot eat, cannot rest, so diary instead. since my rebuff of yesterday I have a sort of empty feeling. Nothing in the world seems of sufficient importance to be worth the doing. As I knew that the only cure for this sort of thing was work, I went amongst the patients. I picked out one who has afforded me a study of much interest. He is so quaint that I am determined to understand him as well as I can. Today I seemed to get nearer than ever before to the heart of his mystery. 

I questioned him more fully than I had ever done, with a view to making myself master of the facts of his hallucination. In my manner of doing it there was, I now see, something of cruelty. I seemed to wish to keep him to the point of his madness, a thing which I avoid with the patients as I would the mouth of hell. 

(Mem., Under what circumstances would I not avoid the pit of hell?) Omnia Romae venalia sunt. Hell has its price! If there be anything behind this instinct it will be valuable to trace it afterwards accurately, so I had better commence to do so, therefore . . . 

R. M, Renfield, age 59. Sanguine temperament, great physical strength, morbidly excitable, periods of gloom, ending in some fixed idea which I cannot make out. I presume that the sanguine temperament itself and the disturbing influence end in a mentally-accomplished finish, a possibly dangerous man, probably dangerous if unselfish. In selfish men caution is as secure an armour for their foes as for themselves. What I think of on this point is, when self is the fixed point the centripetal force is balanced with the centrifugal. When duty, a cause, etc., is the fixed point, the latter force is paramount, and only accident of a series of accidents can balance it. 

LETTER, QUINCEY P. MORRIS TO HON. ARTHUR HOLMOOD 

25 May. 

My dear Art, 

We've told yarns by the campfire in the prairies, and dressed one another's wounds after trying a landing at the Marquesas, and drunk healths on the shore of Titicaca. There are more yarns to be told, and other wounds to be healed, and another health to be drunk. Won't you let this be at my campfire tomorrow night? I have no hesitation in asking you, as I know a certain lady is engaged to a certain dinner party, and that you are free. There will only be one other, our old pal at the Korea, Jack Seward. He's coming, too, and we both want to mingle our weeps over the wine cup, and to drink a health with all our hearts to the happiest man in all the wide world, who has won the noblest heart that God has made and best worth winning. We promise you a hearty welcome, and a loving greeting, and a health as true as your own right hand. We shall both swear to leave you at home if you drink too deep to a certain pair of eyes. Come! 

Yours, as ever and always, 

Quincey P. Morris 

TELEGRAM FROM ARTHUR HOLMWOOD TO QUINCEY P. MORRIS 

26 May 

Count me in every time. I bear messages which will make both your ears tingle. Art 
