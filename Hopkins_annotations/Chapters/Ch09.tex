LETTER, MINA HARKER TO LUCY WESTENRA 

Buda-Pesth, 24 August. 

"My dearest Lucy, 

"I know you will be anxious to hear all that has happened since we parted at the railway station at Whitby. 

"Well, my dear, I got to Hull all right, and caught the boat to Hamburg, and then the train on here. I feel that I can hardly recall anything of the journey, except that I knew I was coming to Jonathan, and that as I should have to do some nursing, I had better get all the sleep I could. I found my dear one, oh, so thin and pale and weaklooking. All the resolution has gone out of his dear eyes, and that quiet dignity which I told you was in his face has vanished. He is only a wreck of himself, and he does not remember anything that has happened to him for a long time past. At least, he wants me to believe so, and I shall never ask. 

"He has had some terrible shock, and I fear it might tax his poor brain if he were to try to recall it. Sister Agatha, who is a good creature and a born nurse, tells me that he wanted her to tell me what they were, but she would only cross herself, and say she would never tell. That the ravings of the sick were the secrets of God, and that if a nurse through her vocation should hear them, she should respect her trust.. 

"She is a sweet, good soul, and the next day, when she saw I was troubled, she opened up the subject my poor dear raved about, added, `I can tell you this much, my dear. That it was not about anything which he has done wrong himself, and you, as his wife to be, have no cause to be concerned. He has not forgotten you or what he owes to you. His fear was of great and terrible things, which no mortal can treat of.' 

"I do believe the dear soul thought I might be jealous lest my poor dear should have fallen in love with any other girl. The idea of my being jealous about Jonathan! And yet, my dear, let me whisper, I felt a thrill of joy through me when I knew that no other woman was a cause for trouble. I am now sitting by his bedside, where I can see his face while he sleeps. He is waking! 

"When he woke he asked me for his coat, as he wanted to get something from the pocket. I asked Sister Agatha, and she brought all his things. I saw amongst them was his notebook, and was was going to ask him to let me look at it, for I knew that I might find some clue to his trouble, but I suppose he must have seen my wish in my eyes, for he sent me over to the window, saying he wanted to be quite alone for a moment. 

"Then he called me back, and he said to me very solemnly, `Wilhelmina', I knew then that he was in deadly earnest, for he has never called me by that name since he asked me to marry him, `You know, dear, my ideas of the trust between husband and wife. There should be no secret, no concealment. I have had a great shock, and when I try to think of what it is I feel my head spin round, and I do not know if it was real of the dreaming of a madman. You know I had brain fever, and that is to be mad. The secret is here, and I do not want to know it. I want to take up my life here, with our marriage.' For, my dear, we had decided to be married as soon as the formalities are complete. `Are you willing, Wilhelmina, to share my ignorance? Here is the book. Take it and keep it, read it if you will,but never let me know unless, indeed, some solemn duty should come upon me to go back to the bitter hours, asleep or awake, sane or mad, recorded here.' He fell back exhausted, and I put the book under his pillow, and kissed him. have asked Sister Agatha to beg the Superior to let our wedding be this afternoon, and am waiting her reply . . ." 

"She has come and told me that the Chaplain of the English mission church has been sent for. We are to be married in an hour, or as soon after as Jonathan awakes." 

"Lucy, the time has come and gone. I feel very solemn, but very, very happy. Jonathan woke a little after the hour, and all was ready, and he sat up in bed, propped up with pillows. He answered his `I will' firmly and strong. I could hardly speak. My heart was so full that even those words seemed to choke me. 

"The dear sisters were so kind. Please, God, I shall never, never forget them, nor the grave and sweet responsibilities I have taken upon me. I must tell you of my wedding present. When the chaplain and the sisters had left me alone with my husband--oh, Lucy, it is the first time I have written the words `my husband'--left me alone with my husband, I took the book from under his pillow, and wrapped it up in white paper, and tied it with a little bit of pale blue ribbon which was round my neck, and sealed it over the knot with sealing wax, and for my seal I used my wedding ring. Then I kissed it and showed it to my husband, and told him that I would keep it so, and then it would be an outward and visible sign for us all our lives that we trusted each other, that I would never open it unless it were for his own dear sake or for the sake of some stern duty. Then he took my hand in his, and oh, Lucy, it was the first time he took his wifes' hand, and said that it was the dearest thing in all the wide world, and that he would go through all the past again to win it, if need be. The poor dear meant to have said a part of the past, but he cannot think of time yet, and I shall not wonder if at first he mixes up not only the month, but the year. 

"Well, my dear, could I say? I could only tell him that I was the happiest woman in all the wide world, and that I had nothing to give him except myself, my life, and my trust, and that with these went my love and duty for all the days of my life. And, my dear, when he kissed me, and drew me to him with his poor weak hands, it was like a solemn pledge between us. 

"Lucy dear, do you know why I tell you all this? It is not only because it is all sweet to me, but because you have been, and are, very dear to me. It was my privilege to be your friend and guide when you came from the schoolroom to prepare for the world of life. I want you to see now, and with the eyes of a very happy wife, whither duty has led me, so that in your own married life you too may be all happy, as I am. My dear, please Almighty God, your life may be all it promises, a long day of sunshine, with no harsh wind, no forgetting duty, no distrust. I must not wish you no pain, for that can never be, but I do hope you will be always as happy as I am now. Goodbye, my dear. I shall post this at once, and perhaps, write you very soon again. I must stop, for Jonathan is waking. I must attend my husband! "Your ever-loving "Mina Harker." 

LETTER, LUCY WESTENRA TO MINA HARKER. 

Whitby, 30 August. 

"My dearest Mina, 

"Oceans of love and millions of kisses, and may you soon be in your own home with your husband. I wish you were coming home soon enough to stay with us here. The strong air would soon restore Jonathan. It has quite restored me. I have an appetite like a cormorant, am full of life, and sleep well. You will be glad to know that I have quite given up walking in my sleep. I think I have not stirred out of my bed for a week, that is when I once got into it at night. Arthur says I am getting fat. By the way, I forgot to tell you that Arthur is here. We have such walks and drives, and rides, and rowing, and tennis, and fishing together, and I love him more than ever. He tells me that he loves me more, but I doubt that, for at first he told me that he couldn't love me more than he did then. But this is nonsense. There he is, calling to me. So no more just at present from your loving, "Lucy. 

"P. S.--Mother sends her love. She seems better, poor dear. 

"P. P.S.--We are to be married on 28 September." 

DR. SEWARDS DIARY 

20 August.--The case of Renfield grows even more interesting. He has now so far quieted that there are spells of cessation from his passion. For the first week after his attack he was perpetually violent. Then one night, just as the moon rose, he grew quiet, and kept murmuring to himself. "Now I can wait. Now I can wait." 

The attendant came to tell me, so I ran down at once to have a look at him. He was still in the strait waistcoat and in the padded room, but the suffused look had gone from his face, and his eyes had something of their old pleading. I might almost say, cringing, softness. I was satisfied with his present condition, and directed him to be relieved. The attendants hesitated, but finally carried out my wishes without protest. 

It was a strange thing that the patient had humour enough to see their distrust, for, coming close to me, he said in a whisper, all the while looking furtively at them, "They think I could hurt you! Fancy me hurting you! The fools!" 

It was soothing, somehow, to the feelings to find myself disassociated even in the mind of this poor madman from the others, but all the same I do not follow his thought. Am I to take it that I have anything in common with him, so that we are, as it were, to stand together. Or has he to gain from me some good so stupendous that my well being is needful to Him? I must find out later on. Tonight he will not speak. Even the offer of a kitten or even a full-grown cat will not tempt him. 

He will only say, "I don't take any stock in cats. I have more to think of now, and I can wait. I can wait." 

After a while I left him. The attendant tells me that he was quiet until just before dawn, and that then he began to get uneasy, and at length violent, until at last he fell into a paroxysm which exhausted him so that he swooned into a sort of coma. 

. . . Three nights has the same thing happened, violent all day then quiet from moonrise to sunrise. I wish I could get some clue to the cause. It would almost seem as if there was some influence which came and went. Happy thought! We shall tonight play sane wits against mad ones. He escaped before without our help. Tonight he shall escape with it. We shall give him a chance, and have the men ready to follow in case they are required. 

23 August.--"The expected always happens." How well Disraeli knew life. Our bird when he found the cage open would not fly, so all our subtle arrangements were for nought. At any rate, we have proved one thing, that the spells of quietness last a reasonable time. We shall in future be able to ease his bonds for a few hours each day. I have given orders to the night attendant merely to shut him in the padded room, when once he is quiet, until the hour before sunrise. The poor soul's body will enjoy the relief even if his mind cannot appreciate it. Hark! The unexpected again! I am called. The patient has once more escaped. 

Later.--Another night adventure. Renfield artfully waited until the attendant was entering the room to inspect. Then he dashed out past him and flew down the passage. I sent word for the attendants to follow. Again he went into the grounds of the deserted house, and we found him in the same place, pressed against the old chapel door. When he saw me he became furious, and had not the attendants seized him in time, he would have tried to kill me. As we sere holding him a strange thing happened. He suddenly redoubled his efforts, and then as suddenly grew calm. I looked round instinctively, but could see nothing. Then I caught the patient's eye and followed it, but could trace nothing as it looked into the moonlight sky, except a big bat, which was flapping its silent and ghostly way to the west. Bats usually wheel about, but this one seemed to go straight on, as if it knew where it was bound for or had some intention of its own. 

The patient grew calmer every instant, and presently said, "You needn't tie me. I shall go quietly!" Without trouble, we came back to the house. I feel there is something ominous in his calm, and shall not forget this night. 

LUCY WESTENRA'S DIARY 

Hillingham, 24 August.--I must imitate Mina, and keep writing things down. Then we can have long talks when we do meet. I wonder when it will be. I wish she were with me again, for I feel so unhappy. Last night I seemed to be dreaming again just as I was at Whitby. Perhaps it is the change of air, or getting home again. It is all dark and horrid to me, for I can remember nothing. But I am full of vague fear, and I feel so weak and worn out. When Arthur came to lunch he looked quite grieved when he saw me, and I hadn't the spirit to try to be cheerful. I wonder if I could sleep in mother's room tonight. I shall make an excuse to try. 

25 August.--Another bad night. Mother did not seem to take to my proposal. She seems not too well herself, and doubtless she fears to worry me. I tried to keep awake, and succeeded for a while, but when the clock struck twelve it waked me from a doze, so I must have been falling asleep. There was a sort of scratching or flapping at the window, but I did not mind it, and as I remember no more, I suppose I must have fallen asleep. More bad dreams. I wish I could remember them. This morning I am horribly weak. My face is ghastly pale, and my throat pains me. It must be something wrong with my lungs, for I don't seem to be getting air enough. I shall try to cheer up when Arthur comes, or else I know he will be miserable to see me so. 

LETTER, ARTHUR TO DR. SEWARD 

"Albemarle Hotel, 31 August "My dear Jack, 

"I want you to do me a favour. Lucy is ill, that is she has no special disease, but she looks awful, and is getting worse every day. I have asked her if there is any cause, I not dare to ask her mother, for to disturb the poor lady's mind about her daughter in her present state of health would be fatal. Mrs. Westenra has confided to me that her doom is spoken, disease of the heart, though poor Lucy does not know it yet. I am sure that there is something preying on my dear girl's mind. I am almost distracted when I think of her. To look at her gives me a pang. I told her I should ask you to see her, and though she demurred at first, I know why, old fellow, she finally consented. It will be a painful task for you, I know, old friend, but it is for her sake, and I must not hesitate to ask, or you to act. You are to come to lunch at Hillingham tomorrow, two o'clock, so as not to arouse any suspicion in Mrs. Westenra, and after lunch Lucy will take an opportunity of being alone with you. I am filled with anxiety, and want to consult with you alone as soon as I can after you have seen her. Do not fail! "Arthur." TELEGRAM, ARTHUR HOLMWOOD TO SEWARD 

1 September 

"Am summoned to see my father, who is worse. Am writing. Write me fully by tonight's post to Ring. Wire me if necessary." 

LETTER FROM DR. SEWARD TO ARTHUR HOLMWOOD 

2 September 

"My dear old fellow, 

"With regard to Miss Westenra's health I hasten to let you know at once that in my opinion there is not any functal disturbance or any malady that I know of. At the same time, I am not by any means satisfied with her appearance. She is woefully different from what she was when I saw her last. Of course you must bear in mind that I did not have full opportunity of examination such as I should wish. Our very friendship makes a little difficulty which not even medical science or custom can bridge over. I had better tell you exactly what happened, leaving you to draw, in a measure, your own conclusions. I shall then say what I have done and propose doing. 

"I found Miss Westenra in seemingly gay spirits. Her mother was present, and in a few seconds I made up my mind that she was trying all she knew to mislead her mother and prevent her from being anxious. I have no doubt she guesses, if she does not know, what need of caution there is. 

"We lunched alone, and as we all exerted ourselves to be cheerful, we got, as some kind of reward for our labours, some real cheerfulness amongst us. Then Mrs. Westenra went to lie down, and Lucy was left with me. We went into her boudoir, and till we got there her gaiety remained, for the servants were coming and going. 

"As soon as the door was closed, however, the mask fell from her face, and she sank down into a chair with a great sigh, and hid her eyes with her hand. When I saw that her high spirits had failed, I at once took advantage of her reaction to make a diagnosis. 

"She said to me very sweetly, `I cannot tell you how I loathe talking about myself.' I reminded her that a doctor's confidence was sacred, but that you were grievously anxious about her. She caught on to my meaning at once, and settled that matter in a word. `Tell Arthur everything you choose. I do not care for myself, but for him!' So I am quite free. 

"I could easily see that she was somewhat bloodless, but I could not see the usual anemic signs, and by the chance ,I was able to test the actual quality of her blood, for in opening a window which was stiff a cord gave way, and she cut her hand slightly with broken glass. It was a slight matter in itself, but it gave me an evident chance, and I secured a few drops of the blood and have analysed them. 

"The qualitative analysis give a quite normal condition, and shows, I should infer, in itself a vigorous state of health. In other physical matters I was quite satisfied that there is no need for anxiety, but as there must be a cause somewhere, I have come to the conclusion that it must be something mental. 

"She complains of difficulty breathing satisfactorily at times, and of heavy, lethargic sleep, with dreams that frighten her, but regarding which she can remember nothing. She says that as a child, she used to walk in her sleep, and that when in Whitby the habit came back, and that once she walked out in the night and went to East Cliff, where Miss Murray found her. But she assures me that of late the habit has not returned. 

"I am in doubt, and so have done the best thing I know of. I have written to my old friend and master, Professor Van Helsing, of Amsterdam, who knows as much about obscure diseases as any one in the world. I have asked him to come over, and as you told me that all things were to be at your charge, I have mentioned to him who you are and your relations to Miss Westenra. This, my dear fellow, is in obedience to your wishes, for I am only too proud and happy to do anything I can for her. 

"Van Helsing would, I know, do anything for me for a personal reason, so no matter on what ground he comes, we must accept his wishes. He is a seemingly arbitrary man, this is because he knows what he is talking about better than any one else. He is a philosopher and a metaphysician, and one of the most advanced scientists of his day, and he has, I believe, an absolutely open mind. This, with an iron nerve, a temper of the ice-brook, and indomitable resolution, self-command, and toleration exalted from virtues to blessings, and the kindliest and truest heart that beats, these form his equipment for the noble work that he is doing for mankind, work both in theory and practice, for his views are as wide as his all-embracing sympathy. I tell you these facts that you may know why I have such confidence in him. I have asked him to come at once. I shall see Miss Westenra tomorrow again. She is to meet me at the Stores, so that I may not alarm her mother by too early a repetition of my call. 

"Yours always." 

John Seward 

LETTER, ABRAHAM VAN HELSING, MD, DPh, D. LiT, ETC, ETC, TO DR. SEWARD 

2 September. 

"My good Friend, 

"When I received your letter I am already coming to you. By good fortune I can leave just at once, without wrong to any of those who have trusted me. Were fortune other, then it were bad for those who have trusted, for I come to my friend when he call me to aid those he holds dear. Tell your friend that when that time you suck from my wound so swiftly the poison of the gangrene from that knife that our other friend, too nervous, let slip, you did more for him when he wants my aids and you call for them than all his great fortune could do. But it is pleasure added to do for him, your friend, it is to you that I come. Have near at hand, and please it so arrange that we may see the young lady not too late on tomorrow, for it is likely that I may have to return here that night. But if need be I shall come again in three days, and stay longer if it must. Till then goodbye, my friend John. 

"Van Helsing." 

LETTER, DR. SEWARD TO HON. ARTHUR HOLMWOOD 

3 September 

"My dear Art, 

"Van Helsing has come and gone. He came on with me to Hillingham, and found that, by Lucy's discretion, her mother was lunching out, so that we were alone with her. 

"Van Helsing made a very careful examination of the patient. He is to report to me, and I shall advise you, for of course I was not present all the time. He is, I fear, much concerned, but says he must think. When I told him of our friendship and how you trust to me in the matter, he said, `You must tell him all you think. Tell him him what I think, if you can guess it, if you will. Nay, I am not jesting. This is no jest, but life and death, perhaps more.' I asked what he meant by that, for he was very serious. This was when we had come back to town, and he was having a cup of tea before starting on his return to Amsterdam. He would not give me any further clue. You must not be angry with me, Art, because his very reticence means that all his brains are working for her good. He will speak plainly enough when the time comes, be sure. So I told him I would simply write an account of our visit, just as if I were doing a descriptive special article for THE DAILY TELEGRAPH. He seemed not to notice, but remarked that the smuts of London were not quite so bad as they used to be when he was a student here. I am to get his report tomorrow if he can possibly make it. In any case I am to have a letter. 

"Well, as to the visit, Lucy was more cheerful than on the day I first saw her, and certainly looked better. She had lost something of the ghastly look that so upset you, and her breathing was normal. She was very sweet to the Professor (as she always is),and tried to make him feel at ease, though I could see the poor girl was making a hard struggle for it. 

"I believe Van Helsing saw it, too, for I saw the quick look under his bushy brows that I knew of old. Then he began to chat of all things except ourselves and diseases and with such an infinite geniality that I could see poor Lucy's pretense of animation merge into reality. Then, without any seeming change, he brought the conversation gently round to his visit, and sauvely said, 

"`My dear young miss, I have the so great pleasure because you are so much beloved. That is much, my dear, even were there that which I do not see. They told me you were down in the spirit, and that you were of a ghastly pale. To them I say "Pouf!" ' And he snapped his fingers at me and went on. `But you and I shall show them how wrong they are. How can he', and he pointed at me with the same look and gesture as that with which he pointed me out in his class, on, or rather after, a particular occasion which he never fails to remind me of, `know anything of a young ladies? He has his madmen to play with, and to bring them back to happiness, and to those that love them. It is much to do, and, oh, but there are rewards in that we can bestow such happiness. But the young ladies! He has no wife nor daughter, and the young do not tell themselves to the young, but to the old, like me, who have known so many sorrows and the causes of them. So, my dear, we will send him away to smoke the cigarette in the garden, whiles you and I have little talk all to ourselves.' I took the hint, and strolled about, and presently the professor came to the window and called me in. He looked grave, but said, ` I have made careful examination, but there is no functional cause. With you I agree that there has been much blood lost, it has been but is not. But the conditions of her are in no way anemic. I have asked her to send me her maid, that I may ask just one or two questions, that so I may not chance to miss nothing. I know well what she will say. And yet there is cause. There is always cause for everything. I must go back home and think. You must send me the telegram every day, and if there be cause I shall come again. The disease, for not to be well is a disease, interest me, and the sweet, young dear, she interest me too. She charm me, and for her, if not for you or disease, I come.' 

"As I tell you, he would not say a word more, even when we were alone. And so now, Art, you know all I know. I shall keep stern watch. I trust your poor father is rallying. It must be a terrible thing to you, my dear old fellow, to be placed in such a position between two people who are both so dear to you. I know your idea of duty to your father, and you are right to stick to it. But if need be, I shall send you word to come at once to Lucy, so do not be over-anxious unless you hear from me." 

DR. SEWARD'S DIARY 

4 September.--Zoophagous patient still keeps up our interest in him. He had only one outburst and that was yesterday at an unusual time. Just before the stroke of noon he began to grow restless. The attendant knew the symptoms, and at once summoned aid. Fortunately the men came at a run, and were just in time, for at the stroke of noon he became so violent that it took all their strength to hold him. In about five minutes, however, he began to get more quiet,and finally sank into a sort of melancholy, in which state he has remained up to now. The attendant tells me that his screams whilst in the paroxysm were really appalling. I found my hands full when I got in, attending to some of the other patients who were frightened by him. Indeed, I can quite understand the effect, for the sounds disturbed even me, though I was some distance away. It is now after the dinner hour of the asylum, and as yet my patient sits in a corner brooding, with a dull, sullen, woe-begone look in his face, which seems rather to indicate than to show something directly. I cannot quite understand it. 

Later.--Another change in my patient. At five o'clock I looked in on him, and found him seemingly as happy and contented as he used to be. He was catching flies and eating them, and was keeping note of his capture by making nailmarks on the edge of the door between the ridges of padding. When he saw me, he came over and apologized for his bad conduct, and asked me in a very humble, cringing way to be led back to his own room, and to have his notebook again. I thought it well to humour him, so he is back in his room with the window open. He has the sugar of his tea spread out on the window sill, and is reaping quite a harvest of flies. He is not now eating them, but putting them into a box, as of old, and is already examining the corners of his room to find a spider. I tried to get him to talk about the past few days, for any clue to his thoughts would be of immense help to me, but he would not rise. For a moment or two he looked very sad, and said in a sort of far away voice, as though saying it rather to himself than to me. 

"All over! All over! He has deserted me. No hope for me now unless I do it myself!" Then suddenly turning to me in a resolute way, he said,"Doctor, won't you be very good to me and let me have a little more sugar? I think it would be very good for me." 

"And the flies?" I said. 

"Yes! The flies like it, too, and I like the flies, therefore I like it."And there are people who know so little as to think that madmen do not argue. I procured him a double supply, and left him as happy a man as, I suppose, any in the world. I wish I could fathom his mind. 

Midnight.--Another change in him. I had been to see Miss Westenra, whom I found much better, and had just returned, and was standing at our own gate looking at the sunset, when once more I heard him yelling. As his room is on this side of the house, I could hear it better than in the morning. It was a shock to me to turn from the wonderful smoky beauty of a sunset over London, with its lurid lights and inky shadows and all the marvellous tints that come on foul clouds even as on foul water, and to realize all the grim sternness of my own cold stone building, with its wealth of breathing misery, and my own desolate heart to endure it all. I reached him just as the sun was going down, and from his window saw the red disc sink. As it sank he became less and less frenzied, and just as it dipped he slid from the hands that held him, an inert mass, on the floor. It is wonderful, however, what intellectual recuperative power lunatics have, for within a few minutes he stood up quite calmly and looked around him. I signalled to the attendants not to hold him, for I was anxious to see what he would do. He went straight over to the window and brushed out the crumbs of sugar. Then he took his fly box, and emptied it outside, and threw away the box. Then he shut the window, and crossing over, sat down on his bed. All this surprised me, so I asked him,"Are you going to keep flies any more?" 

"No," said he. "I am sick of all that rubbish!" He certainly is a wonderfully interesting study. I wish I could get some glimpse of his mind or of the cause of his sudden passion. Stop. There may be a clue after all, if we can find why today his paroxysms came on at high noon and at sunset. Can it be that there is a malign influence of the sun at periods which affects certain natures, as at times the moon does others? We shall see. 

TELEGRAM. SEWARD, LONDON, TO VAN HELSING, AMSTERDAM 

"4 September.--Patient still better today." 

TELEGRAM, SEWARD, LONDON, TO VAN HELSING, AMSTERDAM 

"5 September.--Patient greatly improved. Good appetite, sleeps naturally, good spirits, color coming back." 

TELEGRAM, SEWARD, LONDON, TO VAN HELSING, AMSTERDAM 

"6 September.--Terrible change for the worse. Come at once. Do not lose an hour. I hold over telegram to Holmwood till have seen you." 
