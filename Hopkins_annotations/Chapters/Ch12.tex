\chapter{DR. SEWARD'S DIARY}

18 September.--I drove at once to Hillingham and arrived early. Keeping my cab at the gate, I went up the avenue alone. I knocked gently and rang as quietly as possible, for I feared to disturb Lucy or her mother, and hoped to only bring a servant to the door. After a while, finding no response, I knocked and rang again, still no answer. I cursed the laziness of the servants that they should lie abed at such an hour, for it was now ten o'clock, and so rang and knocked again, but more impatiently, but still without response. Hitherto I had blamed only the servants, but now a terrible fear began to assail me. Was this desolation but another link in the chain of doom which seemed drawing tight round us? Was it indeed a house of death to which I had come, too late? I know that minutes, even seconds of delay, might mean hours of danger to Lucy, if she had had again one of those frightful relapses, and I went round the house to try if I could find by chance an entry anywhere. I could find no means of ingress. Every window and door was fastened and locked, and I returned baffled to the porch. As I did so, I heard the rapid pit-pat of a swiftly driven horse's feet. They stopped at the gate, and a few seconds later I met Van Helsing running up the avenue. When he saw me, he gasped out, "Then it was you, and just arrived. How is she? Are we too late? Did you not get my telegram?" 

I answered as quickly and coherently as I could that I had only got his telegram early in the morning, and had not a minute in coming here, and that I could not make any one in the house hear me. He paused and raised his hat as he said solemnly, "Then I fear we are too late. God's will be done!" 

With his usual recuperative energy, he went on, "Come. If there be no way open to get in, we must make one. Time is all in all to us now." 

We went round to the back of the house, where there was a kitchen window. The Professor took a small surgical saw from his case, and handing it to me, pointed to the iron bars which guarded the window. I attacked them at once and had very soon cut through three of them. Then with a long, thin knife we pushed back the fastening of the sashes and opened the window. I helped the Professor in, and followed him. There was no one in the kitchen or in the servants' rooms, which were close at hand. We tried all the rooms as we went along, and in the dining room, dimly lit by rays of light through the shutters, found four servant women lying on the floor. There was no need to think them dead, for their stertorous breathing and the acrid smell of laudanum in the room left no doubt as to their condition. 

Van Helsing and I looked at each other, and as we moved away he said, "We can attend to them later."Then we ascended to Lucy's room. For an instant or two we paused at the door to listen, but there was no sound that we could hear. With white faces and trembling hands, we opened the door gently, and entered the room. 

How shall I describe what we saw? On the bed lay two women, Lucy and her mother. The latter lay farthest in, and she was covered with a white sheet, the edge of which had been blown back by the drought through the broken window, showing the drawn, white, face, with a look of terror fixed upon it. By her side lay Lucy, with face white and still more drawn. The flowers which had been round her neck we found upon her mother's bosom, and her throat was bare, showing the two little wounds which we had noticed before, but looking horribly white and mangled. Without a word the Professor bent over the bed, his head almost touching poor Lucy's breast. Then he gave a quick turn of his head, as of one who listens, and leaping to his feet, he cried out to me, "It is not yet too late! Quick! Quick! Bring the brandy!" 

I flew downstairs and returned with it, taking care to smell and taste it, lest it, too, were drugged like the decanter of sherry which I found on the table. The maids were still breathing, but more restlessly, and I fancied that the narcotic was wearing off. I did not stay to make sure, but returned to Van Helsing. He rubbed the brandy, as on another occasion, on her lips and gums and on her wrists and the palms of her hands. He said to me, "I can do this, all that can be at the present. You go wake those maids. Flick them in the face with a wet towel, and flick them hard. Make them get heat and fire and a warm bath. This poor soul is nearly as cold as that beside her. She will need be heated before we can do anything more." 

I went at once, and found little difficulty in waking three of the women. The fourth was only a young girl, and the drug had evidently affected her more strongly so I lifted her on the sofa and let her sleep. 

The others were dazed at first, but as remembrance came back to them they cried and sobbed in a hysterical manner. I was stern with them, however, and would not let them talk. I told them that one life was bad enough to lose, and if they delayed they would sacrifice Miss Lucy. So, sobbing and crying they went about their way, half clad as they were, and prepared fire and water. Fortunately, the kitchen and boiler fires were still alive, and there was no lack of hot water. We got a bath and carried Lucy out as she was and placed her in it. Whilst we were busy chafing her limbs there was a knock at the hall door. One of the maids ran off, hurried on some more clothes, and opened it. Then she returned and whispered to us that there was a gentleman who had come with a message from Mr. Holmwood. I bade her simply tell him that he must wait, for we could see no one now. She went away with the message, and, engrossed with our work, I clean forgot all about him. 

I never saw in all my experience the Professor work in such deadly earnest. I knew, as he knew, that it was a stand-up fight with death, and in a pause told him so. He answered me in a way that I did not understand, but with the sternest look that his face could wear. 

"If that were all, I would stop here where we are now, and let her fade away into peace, for I see no light in life over her horizon." He went on with his work with, if possible, renewed and more frenzied vigour. 

Presently we both began to be conscious that the heat was beginning to be of some effect. Lucy's heart beat a trifle more audibly to the stethoscope, and her lungs had a perceptible movement. Van Helsing's face almost beamed, and as we lifted her from the bath and rolled her in a hot sheet to dry her he said to me, "The first gain is ours! Check to the King!" 

We took Lucy into another room, which had by now been prepared, and laid her in bed and forced a few drops of brandy down her throat. I noticed that Van Helsing tied a soft silk handkerchief round her throat. She was still unconscious, and was quite as bad as, if not worse than, we had ever seen her. 

Van Helsing called in one of the women, and told her to stay with her and not to take her eyes off her till we returned, and then beckoned me out of the room. 

"We must consult as to what is to be done," he said as we descended the stairs. In the hall he opened the dining room door, and we passed in, he closing the door carefully behind him. The shutters had been opened, but the blinds were already down, with that obedience to the etiquette of death which the British woman of the lower classes always rigidly observes. The room was, therefore, dimly dark. It was, however, light enough for our purposes. Van Helsing's sternness was somewhat relieved by a look of perplexity. He was evidently torturing his mind about something, so I waited for an instant, and he spoke. 

"What are we to do now? Where are we to turn for help? We must have another transfusion of blood, and that soon, or that poor girl's life won't be worth an hour's purchase. You are exhausted already. I am exhausted too. I fear to trust those women, even if they would have courage to submit. What are we to do for some one who will open his veins for her?" 

"What's the matter with me, anyhow?" 

The voice came from the sofa across the room, and its tones brought relief and joy to my heart, for they were those of Quincey Morris. 

Van Helsing started angrily at the first sound, but his face softened and a glad look came into his eyes as I cried out, "Quincey Morris!" and rushed towards him with outstretched hands. 

"What brought you her?" I cried as our hands met. 

"I guess Art is the cause." 

He handed me a telegram.-- `Have not heard from Seward for three days, and am terribly anxious. Cannot leave. Father still in same condition. Send me word how Lucy is. Do not delay.--Holmwood.' 

"I think I came just in the nick of time. You know you have only to tell me what to do." 

Van Helsing strode forward, and took his hand, looking him straight in the eyes as he said, "A brave man's blood is the best thing on this earth when a woman is in trouble. You're a man and no mistake. Well, the devil may work against us for all he's worth, but God sends us men when we want them." 

Once again we went through that ghastly operation. I have not the heart to go through with the details. Lucy had got a terrible shock and it told on her more than before, for though plenty of blood went into her veins, her body did not respond to the treatment as well as on the other occasions. Her struggle back into life was something frightful to see and hear. However, the action of both heart and lungs improved, and Van Helsing made a sub-cutaneous injection of morphia, as before, and with good effect. Her faint became a profound slumber. The Professor watched whilst I went downstairs with Quincey Morris, and sent one of the maids to pay off one of the cabmen who were waiting. 

I left Quincey lying down after having a glass of wine, and told the cook to get ready a good breakfast. Then a thought struck me, and I went back to the room where Lucy now was. When I came softly in, I found Van Helsing with a sheet or two of note paper in his hand. He had evidently read it, and was thinking it over as he sat with his hand to his brow. There was a look of grim satisfaction in his face, as of one who has had a doubt solved. He handed me the paper saying only, "It dropped from Lucy's breast when we carried her to the bath." 

When I had read it, I stook looking at the Professor, and after a pause asked him, "In God's name, what does it all mean? Was she, or is she, mad, or what sort of horrible danger is it?" I was so bewildered that I did not know what to say more. Van Helsing put out his hand and took the paper, saying, 

"Do not trouble about it now. Forget if for the present. You shall know and understand it all in good time, but it will be later. And now what is it that you came to me to say?" This brought me back to fact, and I was all myself again. 

"I came to speak about the certificate of death. If we do not act properly and wisely, there may be an inquest, and that paper would have to be produced. I am in hopes that we need have no inquest, for if we had it would surely kill poor Lucy, if nothing else did. I know, and you know, and the other doctor who attended her knows, that Mrs. Westenra had disease of the heart, and we can certify that she died of it. Let us fill up the certificate at once, and I shall take it myself to the registrar and go on to the undertaker." 

"Good, oh my friend John! Well thought of! Truly Miss Lucy, if she be sad in the foes that beset her, is at least happy in the friends thatlove her. One, two, three, all open their veins for her, besides one old man. Ah, yes, I know, friend John. I am not blind! I love you all the more for it! Now go." 

In the hall I met Quincey Morris, with a telegram for Arthur telling him that Mrs. Westenra was dead, that Lucy also had been ill, but was now going on better, and that Van Helsing and I were with her. I told him where I was going, and he hurried me out, but as I was going said, "When you come back, Jack, may I have two words with you all to ourselves?" I nodded in reply and went out. I found no difficulty about the registration, and arranged with the local undertaker to come up in the evening to measure for the coffin and to make arrangements. 

When I got back Quincey was waiting for me. I told him I would see him as soon as I knew about Lucy, and went up to her room. She was still sleeping, and the Professor seemingly had not moved from his seat at her side. From his putting his finger to his lips, I gathered that he expected her to wake before long and was afraid of fore-stalling nature. So I went down to Quincey and took him into the breakfast room, where the blinds were not drawn down, and which was a little more cheerful, or rather less cheerless, than the other rooms. 

When we were alone, he said to me, "Jack Seward, I don't want to shove myself in anywhere where I've no right to be, but this is no ordinary case. You know I loved that girl and wanted to marry her, but although that's all past and gone, I can't help feeling anxious about her all the same. What is it that's wrong with her? The Dutchman, and a fine old fellow is is, I can see that, said that time you two came into the room, that you must have another transfusion of blood, and that both you and he were exhausted. Now I know well that you medical men speak in camera, and that a man must not expect to know what they consult about in private. But this is no common matter, and whatever it is, I have done my part. Is not that so?" 

"That's so," I said, and he went on. 

"I take it that both you and Van Helsing had done already what I did today. Is not that so?" 

"That's so." 

"And I guess Art was in it too. When I saw him four days ago down at his own place he looked queer. I have not seen anything pulled down so quick since I was on the Pampas and had a mare that I was fond of go to grass all in a night. One of those big bats that they call vampires had got at her in the night, and what with his gorge and the vein left open, there wasn't enough blood in her to let her stand up, and I had to put a bullet through her as she lay. Jack, if you may tell me without betraying confidence, Arthur was the first, is not that so?" 

As he spoke the poor fellow looked terribly anxious. He was in a torture of suspense regarding the woman he loved, and his utter ignorance of the terrible mystery which seemed to surround her intensified his pain. His very heart was bleeding, and it took all the manhood of him, and there was a royal lot of it, too, to keep him from breaking down. I paused before answering, for I felt that I must not betray anything which the Professor wished kept secret, but already he knew so much, and guessed so much, that there could be no reason for not answering, so I answered in the same phrase. 

"That's so." 

"And how long has this been going on?" 

"About ten days." 

"Ten days! Then I guess, Jack Seward, that that poor pretty creature that we all love has had put into her veins within that time the blood of four strong men. Man alive, her whole body wouldn't hold it." Then coming close to me, he spoke in a fierce half-whisper. "What took it out?" 

I shook my head. "That," I said, "is the crux. Van Helsing is simply frantic about it, and I am at my wits' end. I can't even hazard a guess. There has been a series of little circumstances which have thrown out all our calculations as to Lucy being properly watched. But these shall not occur again. Here we stay until all be well, or ill." 

Quincey held out his hand. "Count me in," he said. "You and the Dutchman will tell me what to do, and I'll do it." 

When she woke late in the afternoon, Lucy's first movement was to feel in her breast, and to my surprise, produced the paper which Van Helsing had given me to read. The careful Professor had replaced it where it had come from, lest on waking she should be alarmed. Her eyes then lit on Van Helsing and on me too, and gladdened. Then she looked round the room, and seeing where she was, shuddered. She gave a loud cry, and put her poor thin hands before her pale face. 

We both understood what was meant, that she had realized to the full her mother's death. So we tried what we could to comfort her. Doubtless sympathy eased her somewhat, but she was very low in thought and spirit, and wept silently and weakly for a long time. We told her that either or both of us would now remain with her all the time, and that seemed to comfort her. Towards dusk she fell into a doze. Here a very odd thing occurred. Whilst still asleep she took the paper from her breast and tore it in two. Van Helsing stepped over and took the pieces from her. All the same, however, she went on with the action of tearing, as though the material were still in her hands. Finally she lifted her hands and opened them as though scattering the fragments. Van Helsing seemed surprised, and his brows gathered as if in thought, but he said nothing. 

19 September.--All last night she slept fitfully, being always afraid to sleep, and something weaker when she woke from it. The Professor and I took in turns to watch, and we never left her for a moment unattended. Quincey Morris said nothing about his intention, but I knew that all night long he patrolled round and round the house. 

When the day came, its searching light showed the ravages in poor Lucy's strength. She was hardly able to turn her head, and the little nourishment which she could take seemed to do her no good. At times she slept, and both Van Helsing and I noticed the difference in her, between sleeping and waking. Whilst asleep she looked stronger, although more haggard, and her breathing was softer. Her open mouth showed the pale gums drawn back from the teeth, which looked positively longer and sharper than usual. When she woke the softness of her eyes evidently changed the expression, for she looked her own self, although a dying one. In the afternoon she asked for Arthur, and we telegraphed for him. Quincey went off to meet him at the station. 

When he arrived it was nearly six o'clock, and the sun was setting full and warm, and the red light streamed in through the window and gave more color to the pale cheeks. When he saw her, Arthur was simply choking with emotion, and none of us could speak. In the hours that had passed, the fits of sleep, or the comatose condition that passed for it, had grown more frequent, so that the pauses when conversation was possible were shortened. Arthur's presence, however, seemed to act as a stimulant. She rallied a little, and spoke to him more brightly than she had done since we arrived. He too pulled himself together, and spoke as cheerily as he could, so that the best was made of everything. 

It is now nearly one o'clock, and he and Van Helsing are sitting with her. I am to relieve them in a quarter of an hour, and I am entering this on Lucy's phonograph. Until six o'clock they are to try to rest. I fear that tomorrow will end our watching, for the shock has been too great. The poor child cannot rally. God help us all. 

LETTER MINA HARKER TO LUCY WESTENRA 

(Unopened by her) 

17 September 

My dearest Lucy, 

"It seems an age since I heard from you, or indeed since I wrote. You will pardon me, I know, for all my faults when you have read all my budget of news. Well, I got my husband back all right. When we arrived at Exeter there was a carriage waiting for us, and in it, though he had an attack of gout, Mr. Hawkins. He took us to his house, where there were rooms for us all nice and comfortable, and we dined together. After dinner Mr. Hawkins said, 

" `My dears, I want to drink your health and prosperity, and may every blessing attend you both. I know you both from children, and have, with love and pride, seen you grow up. Now I want you to make your home here with me. I have left to me neither chick nor child. All are gone, and in my will I have left you everything.' I cried, Lucy dear, as Jonathan and the old man clasped hands. Our evening was a very, very happy one. 

"So here we are, installed in this beautiful old house, and from both my bedroom and the drawing room I can see the great elms of the cathedral close, with their great black stems standing out against the old yellow stone of the cathedral, and I can hear the rooks overhead cawing and cawing and chattering and chattering and gossiping all day, after the manner of rooks--and humans. I am busy, I need not tell you, arranging things and housekeeping. Jonathan and Mr. Hawkins are busy all day, for now that Jonathan is a partner, Mr. Hawkins wants to tell him all about the clients. 

"How is your dear mother getting on? I wish I could run up to town for a day or two to see you, dear, but I, dare not go yet, with so much on my shoulders, and Jonathan wants looking after still. He is beginning to put some flesh on his bones again, but he was terribly weakened by the long illness. Even now he sometimes starts out of his sleep in a sudden way and awakes all trembling until I can coax him back to his usual placidity. However, thank God, these occasions grow less frequent as the days go on, and they will in time pass away altogether, I trust. And now I have told you my news, let me ask yours. When are you to be married, and where, and who is to perform the ceremony, and what are you to wear, and is it to be a public or private wedding? Tell me all about it, dear, tell me all about everything, for there is nothing which interests you which will not be dear to me. Jonathan asks me to send his `respectful duty', but I do not think that is good enough from the junior partner of the important firm Hawkins \& Harker. And so, as you love me, and he loves me, and I love you with all the moods and tenses of the verb, I send you simply his `love' instead. Goodbye, my dearest Lucy, and blessings on you." Yours, Mina Harker 

REPORT FROM PATRICK HENNESSEY, MD, MRCSLK, QCPI, ETC, ETC, TO JOHN SEWARD, MD 

20 September 

My dear Sir: 

"In accordance with your wishes, I enclose report of the conditions of everything left in my charge. With regard to patient, Renfield, there is more to say. He has had another outbreak, which might have had a dreadful ending, but which, as it fortunately happened, was unattended with any unhappy results. This afternoon a carrier's cart with two men made a call at the empty house whose grounds abut on ours, the house to which, you will remember, the patient twice ran away. The men stopped at our gate to ask the porter their way, as they were strangers. 

"I was myself looking out of the study window, having a smoke after dinner, and saw one of them come up to the house. As he passed the window of Renfield's room, the patient began to rate him from within, and called him all the foul names he could lay his tongue to. The man, who seemed a decent fellow enough, contented himself by telling him to `shut up for a foul-mouthed beggar',whereon our man accused him of robbing him and wanting to murder him and said that he would hinder him if he were to swing for it. I opened the window and signed to the man not to notice, so he contented himself after looking the place over and making up his mind as to what kind of place he had got to by saying, `Lor' bless yer, sir, I wouldn't mind what was said to me in a bloomin' madhouse. I pity ye and the guv'nor for havin' to live in the house with a wild beast like that.' 

"Then he asked his way civilly enough, and I told him where the gate of the empty house was. He went away followed by threats and curses and revilings from our man. I went down to see if I could make out any cause for his anger, since he is usually such a well-behaved man, and except his violent fits nothing of the kind had ever occurred. I found him, to my astonishment, quite composed and most genial in his manner. I tried to get him to talk of the incident, but he blandly asked me questions as to what I meant, and led me to believe that he was completely oblivious of the affair. It was, I am sorry to say, however, only another instance of his cunning, for within half an hour I heard of him again. This time he had broken out through the window of his room, and was running down the avenue. I called to the attendants to follow me, and ran after him, for I feared he was intent on some mischief. My fear was justified when I saw the same cart which had passed before coming down the road, having on it some great wooden boxes. The men were wiping their foreheads, and were flushed in the face, as if with violent exercise. Before I could get up to him, the patient rushed at them, and pulling one of them off the cart, began to knock his head against the ground. If I had not seized him just at the moment, I believe he would have killed the man there and then. The other fellow jumped down and struck him over the head with the butt end of his heavy whip. It was a horrible blow, but he did not seem to mind it, but seized him also, and struggled with the three of us, pulling us to and fro as if we were kittens. You know I am no lightweight, and the others were both burly men. At first he was silent in his fighting, but as we began to master him, and the attendants were putting a strait waistcoat on him, he began to shout, `I'll frustrate them! They shan't rob me!They shan't murder me by inches! I'll fight for my Lord and Master!'and all sorts of similar incoherent ravings. It was with very considerable difficulty that they got him back to the house and put him in the padded room. One of the attendants, Hardy, had a finger broken. However, I set it all right, and he is going on well. 

"The two carriers were at first loud in their threats of actions for damages, and promised to rain all the penalties of the law on us. Their threats were, however, mingled with some sort of indirect apology for the defeat of the two of them by a feeble madman. They said that if it had not been for the way their strength had been spent in carrying and raising the heavy boxes to the cart they would have made short work of him. They gave as another reason for their defeat the extraordinary state of drouth to which they had been reduced by the dusty nature of their occupation and the reprehensible distance from the scene of their labors of any place of public entertainment. I quite understood their drift, and after a stiff glass of strong grog, or rather more of the same, and with each a sovereign in hand, they made light of the attack, and swore that they would encounter a worse madman any day for the pleasure of meeting so `bloomin' good a bloke' as your correspondent. I took their names and addresses, in case they might be needed. They are as follows: Jack Smollet, of Dudding's Rents, King George's Road, Great Walworth, and Thomas Snelling, Peter Farley's Row, Guide Court, Bethnal Green. They are both in the employment of Harris \& Sons, Moving and Shipment Company, Orange Master's Yard, Soho. 

"I shall report to you any matter of interest occurring here, and shall wire you at once if there is anything of importance. 

"Believe me, dear Sir, 

"Yours faithfully, 

"Patrick Hennessey." 

LETTER, MINA HARKER TO LUCY WESTENRA (Unopened by her) 

18 September 

"My dearest Lucy, 

"Such a sad blow has befallen us. Mr. Hawkins has died very suddenly. Some may not think it so sad for us, but we had both come to so love him that it really seems as though we had lost a father. I never knew either father or mother, so that the dear old man's death is a real blow to me. Jonathan is greatly distressed. It is not only that he feels sorrow, deep sorrow, for the dear,good man who has befriended him all his life, and now at the end has treated him like his own son and left him a fortune which to people of our modest bringing up is wealth beyond the dream of avarice, but Jonathan feels it on another account. He says the amount of responsibility which it puts upon him makes him nervous. He begins to doubt himself. I try to cheer him up, and my belief in him helps him to have a belief in himself. But it is here that the grave shock that he experienced tells upon him the most. Oh, it is too hard that a sweet, simple, noble, strong nature such as his, a nature which enabled him by our dear, good friend's aid to rise from clerk to master in a few years, should be so injured that the very essence of its strength is gone. Forgive me, dear, if I worry you with my troubles in the midst of your own happiness, but Lucy dear, I must tell someone, for the strain of keeping up a brave and cheerful appearance to Jonathan tries me, and I have no one here that I can confide in. I dread coming up to London, as we must do that day after tomorrow, for poor Mr. Hawkins left in his will that he was to be buried in the grave with his father. As there are no relations at all, Jonathan will have to be chief mourner. I shall try to run over to see you, dearest, if only for a few minutes. Forgive me for troubling you. With all blessings, 

"Your loving 

Mina Harker" DR. SEWARD' DIARY 

20 September.--Only resolution and habit can let me make an entry tonight. I am too miserable, too low spirited, too sick of the world and all in it, including life itself, that I would not care if I heard this moment the flapping of the wings of the angel of death. And he has been flapping those grim wings to some purpose of late, Lucy's mother and Arthur's father, and now . . .Let me get on with my work. 

I duly relieved Van Helsing in his watch over Lucy. We wanted Arthur to go to rest also, but he refused at first. It was only when I told him that we should want him to help us during the day, and that we must not all break down for want of rest, lest Lucy should suffer, that he agreed to go. 

Van Helsing was very kind to him. "Come, my child," he said. "Come with me. You are sick and weak, and have had much sorrow and much mental pain, as well as that tax on your strength that we know of. You must not be alone, for to be alone is to be full of fears and alarms. Come to the drawing room, where there is a big fire, and there are two sofas. You shall lie on one, and I on the other, and our sympathy will be comfort to each other, even though we do not speak, and even if we sleep." 

Arthur went off with him, casting back a longing look on Lucy's face, which lay in her pillow, almost whiter than the lawn. She lay quite still, and I looked around the room to see that all was as it should be. I could see that the Professor had carried out in this room, as in the other, his purpose of using the garlic. The whole of the window sashes reeked with it, and round Lucy's neck, over the silk handkerchief which Van Helsing made her keep on, was a rough chaplet of the same odorous flowers. 

Lucy was breathing somewhat stertorously, and her face was at its worst, for the open mouth showed the pale gums. Her teeth, in the dim, uncertain light, seemed longer and sharper than they had been in the morning. In particular, by some trick of the light, the canine teeth looked longer and sharper than the rest. 

I sat down beside her, and presently she moved uneasily. At the same moment there came a sort of dull flapping or buffeting at the window. I went over to it softly, and peeped out by the corner of the blind. There was a full moonlight, and I could see that the noise was made by a great bat, which wheeled around, doubtless attracted by the light, although so dim, and every now and again struck the window with its wings. When I came back to my seat, I found that Lucy had moved slightly, and had torn away the garlic flowers from her throat. I replaced them as well as I could, and sat watching her. 

Presently she woke, and I gave her food, as Van Helsing had prescribed. She took but a little, and that languidly. There did not seem to be with her now the unconscious struggle for life and strength that had hitherto so marked her illness. It struck me as curious that the moment she became conscious she pressed the garlic flowers close to her. It was certainly odd that whenever she got into that lethargic state, with the stertorous breathing, she put the flowers from her, but that when she waked she clutched them close, There was no possibility of making amy mistake about this, for in the long hours that followed, she had many spells of sleeping and waking and repeated both actions many times. 

At six o'clock Van Helsing came to relieve me. Arthur had then fallen into a doze, and he mercifully let him sleep on. When he saw Lucy's face I could hear the sissing indraw of breath, and he said to me in a sharp whisper."Draw up the blind. I want light!" Then he bent down, and, with his face almost touching Lucy's, examined her carefully. He removed the flowers and lifted the silk handkerchief from her throat. As he did so he started back and I could hear his ejaculation, "Mein Gott!" as it was smothered in his throat. I bent over and looked, too, and as I noticed some queer chill came over me. The wounds on the throat had absolutely disappeared. 

For fully five minutes Van Helsing stood looking at her, with his face at its sternest. Then he turned to me and said calmly, "She is dying. It will not be long now. It will be much difference, mark me, whether she dies conscious or in her sleep. Wake that poor boy, and let him come and see the last. He trusts us, and we have promised him." 

I went to the dining room and waked him. He was dazed for a moment, but when he saw the sunlight streaming in through the edges of the shutters he thought he was late, and expressed his fear. I assured him that Lucy was still asleep, but told him as gently as i could that both Van Helsing and I feared that the end was near. He covered his face with his hands, and slid down on his knees by the sofa, where he remained, perhaps a minute, with his head buried, praying, whilst his shoulders shook with grief. I took him by the hand and raised him up. "Come," I said, "my dear old fellow, summon all your fortitude. It will be best and easiest for her." 

When we came into Lucy's room I could see that Van Helsing had, with his usual forethought, been putting matters straight and making everything look as pleasing as possible. He had even brushed Lucy's hair, so that it lay on the pillow in its usual sunny ripples. When we came into the room she opened her eyes, and seeing him, whispered softly, "Arthur! Oh, my love, I am so glad you have come!" 

He was stooping to kiss her, when Van Helsing motioned him back. "No," he whispered, "not yet! Hold her hand, it will comfort her more." 

So Arthur took her hand and knelt beside her, and she looked her best, with all the soft lines matching the angelic beauty of her eyes. Then gradually her eyes closed, and she sank to sleep. For a little bit her breast heaved softly, and her breath came and went like a tired child's. 

And then insensibly there came the strange change which I had noticed in the night. Her breathing grew stertorous, the mouth opened, and the pale gums, drawn back, made the teeth look longer and sharper than ever. In a sort of sleepwaking, vague, unconscious way she opened her eyes, which were now dull and hard at once, and said in a soft,voluptuous voice, such as I had never heard from her lips, "Arthur! Oh, my love, I am so glad you have come! Kiss me!" 

Arthur bent eagerly over to kiss her, but at that instant Van Helsing, who, like me, had been startled by her voice, swooped upon him, and catching him by the neck with both hands, dragged him back with a fury of strength which I never thought he could have possessed, and actually hurled him almost across the room. "Not on your life!" he said, "not for your living soul and hers!" And he stood between them like a lion at bay. 

Arthur was so taken aback that he did not for a moment know what to do or say, and before any impulse of violence could seize him he realized the place and the occasion, and stood silent, waiting. 

I kept my eyes fixed on Lucy, as did Van Helsing, and we saw a spasm as of rage flit like a shadow over her face. The sharp teeth clamped together. Then her eyes closed, and she breathed heavily. 

Very shortly after she opened her eyes in all their softness, and putting out her poor, pale, thin hand, took Van Helsing's great brown one, drawing it close to her, she kissed it. "My true friend," she said, in a faint voice, but with untellable pathos, "My true friend, and his! Oh, guard him, and give me peace!" 

"I swear it!" he said solemnly, kneeling beside her and holding up his hand, as one who registers an oath. Then he turned to Arthur, and said to him, "Come, my child, take her hand in yours, and kiss her on the forehead, and only once." 

Their eyes met instead of their lips, and so they parted. Lucy's eyes closed, and Van Helsing, who had been watching closely, took Arthur's arm, and drew him away. 

And then Lucy's breathing became stertorous again, and all at once it ceased. 

"It is all over," said Van Helsing. "She is dead!" 

I took Arthur by the arm, and led him away to the drawing room, where he sat down, and covered his face with his hands, sobbing in a way that nearly broke me down to see. 

I went back to the room, and found Van Helsing looking at poor Lucy, and his face was sterner than eve. Some change had come over her body. Death had given back part of her beauty, for her brow and cheeks had recovered some of their flowing lines. Even the lips had lost their deadly pallor. It was as if the blood, no longer needed for the working of the heart, had gone to make the harshness of death as little rude as might be. 

"We thought her dying whilst she slept, And sleeping when she died." 

I stood beside Van Helsing, and said, "Ah well, poor girl, there is peace for her at last. It is the end!" 

He turned to me, and said with grave solemnity,"Not so, alas! Not so. It is only the beginning!" 

When I asked him what he meant, he only shook his head and answered, "We can do nothing as yet. Wait and see." 
