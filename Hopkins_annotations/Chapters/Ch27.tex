MINA HARKER'S JOURNAL 

1 November.--All day long we have travelled, and at a good speed. The horses seem to know that they are being kindly treated, for they go willingly their full stage at best speed. We have now had so many changes and find the same thing so constantly that we are encouraged to think that the journey will be an easy one. Dr. Van Helsing is laconic, he tells the farmers that he is hurrying to Bistritz, and pays them well to make the exchange of horses. We get hot soup, or coffee, or tea, and off we go. It is a lovely country. Full of beauties of all imaginable kinds, and the people are brave, and strong, and simple, and seem full of nice qualities. They are very, very superstitious. In the first house where we stopped, when the woman who served us saw the scar on my forehead, she crossed herself and put out two fingers towards me, to keep off the evil eye. I believe they went to the trouble of putting an extra amount of garlic into our food, and I can't abide garlic. Ever since then I have taken care not to take off my hat or veil, and so have escaped their suspicions. We are travelling fast, and as we have no driver with us to carry tales, we go ahead of scandal. But I daresay that fear of the evil eye will follow hard behind us all the way. The Professor seems tireless. All day he would not take any rest, though he made me sleep for a long spell. At sunset time he hypnotized me, and he says I answered as usual,"darkness, lapping water and creaking wood." So our enemy is still on the river. I am afraid to think of Jonathan, but somehow I have now no fear for him, or for myself. I write this whilst we wait in a farmhouse for the horses to be ready. Dr. Van Helsing is sleeping. Poor dear, he looks very tired and old and grey, but his mouth is set as firmly as a conqueror's. Even in his sleep he is intense with resolution. When we have well started I must make him rest whilst I drive. I shall tell him that we have days before us, and he must not break down when most of all his strength will be needed . . . All is ready. We are off shortly. 

2 November, morning.--I was successful, and we took turns driving all night. Now the day is on us, bright though cold. There is a strange heaviness in the air. I say heaviness for want of a better word. I mean that it oppresses us both. It is very cold, and only our warm furs keep us comfortable. At dawn Van Helsing hypnotized me. He says I answered "darkness, creaking wood and roaring water," so the river is changing as they ascend. I do hope that my darling will not run any chance of danger, more than need be, but we are in God's hands. 

2 November, night.--All day long driving. The country gets wilder as we go, and the great spurs of the Carpathians, which at Veresti seemed so far from us and so low on the horizon, now seem to gather round us and tower in front. We both seem in good spirits. I think we make an effort each to cheer the other, in the doing so we cheer ourselves. Dr. Van Helsing says that by morning we shall reach the Borgo Pass. The houses are very few here now, and the Professor says that the last horse we got will have to go on with us, as we may not be able to change. He got two in addition to the two we changed, so that now we have a rude four-in-hand. The dear horses are patient and good, and they give us no trouble. We are not worried with other travellers, and so even I can drive. We shall get to the Pass in daylight. We do not want to arrive before. So we take it easy, and have each a long rest in turn. Oh, what will tomorrow bring to us? We go to seek the place where my poor darling suffered so much. God grant that we may be guided aright, and that He will deign to watch over my husband and those dear to us both, and who are in such deadly peril. As for me, I am not worthy in His sight. Alas! I am unclean to His eyes, and shall be until He may deign to let me stand forth in His sight as one of those who have not incurred His wrath. 

MEMORANDUM BY ABRAHAM VAN HELSING 

4 November.--This to my old and true friend John Seward, M. D., of Purefleet, London, in case I may not see him. It may explain. It is morning, and I write by a fire which all the night I have kept alive, Madam Mina aiding me. It is cold, cold. So cold that the grey heavy sky is full of snow, which when it falls will settle for all winter as the ground is hardening to receive it. It seems to have affected Madam Mina. She has been so heavy of head all day that she was not like herself. She sleeps, and sleeps, and sleeps! She who is usual so alert, have done literally nothing all the day. She even have lost her appetite. She make no entry into her little diary, she who write so faithful at every pause. Something whisper to me that all is not well. However, tonight she is more vif. Her long sleep all day have refresh and restore her, for now she is all sweet and bright as ever. At sunset I try to hypnotize her, but alas! with no effect. The power has grown less and less with each day, and tonight it fail me altogether. Well, God's will be done, whatever it may be, and whithersoever it may lead! 

Now to the historical, for as Madam Mina write not in her stenography, I must, in my cumbrous old fashion, that so each day of us may not go unrecorded. 

We got to the Borgo Pass just after sunrise yesterday morning. When I saw the signs of the dawn I got ready for the hypnotism. We stopped our carriage, and got down so that there might be no disturbance. I made a couch with furs, and Madam Mina, lying down, yield herself as usual, but more slow and more short time than ever, to the hypnotic sleep. As before, came the answer, "darkness and the swirling of water." Then she woke, bright and radiant and we go on our way and soon reach the Pass. At this time and place, she become all on fire with zeal. Some new guiding power be in her manifested, for she point to a road and say, "This is the way." 

"How know you it?" I ask. 

"Of course I know it,' she answer, and with a pause, add, "Have not my Jonathan travelled it and wrote of his travel?" 

At first I think somewhat strange, but soon I see that there be only one such byroad. It is used but little, and very different from the coach road from the Bukovina to Bistritz, which is more wide and hard, and more of use. 

So we came down this road. When we meet other ways, not always were we sure that they were roads at all, for they be neglect and light snow have fallen, the horses know and they only. I give rein to them, and they go on so patient. By and by we find all the things which Jonathan have note in that wonderful diary of him. Then we go on for long, long hours and hours. At the first, I tell Madam Mina to sleep. She try, and she succeed. She sleep all the time, till at the last, I feel myself to suspicious grow, and attempt to wake her. But she sleep on, and I may not wake her though I try. I do not wish to try too hard lest I harm her. For I know that she have suffer much, and sleep at times be all-in-all to her. I think I drowse myself, for all of sudden I feel guilt, as though I have done something. I find myself bolt up, with the reins in my hand, and the good horses go along jog, jog, just as ever. I look down and find Madam Mina still asleep. It is now not far off sunset time, and over the snow the light of the sun flow in big yellow flood, so that we throw great long shadow on where the mountain rise so steep. For we are going up, and up, and all is oh, so wild and rocky, as though it were the end of the world. 

Then I arouse Madam Mina. This time she wake with not much trouble, and then I try to put her to hypnotic sleep. But she sleep not, being as though I were not. Still I try and try, till all at once I find her and myself in dark, so I look round, and find that the sun have gone down. Madam Mina laugh, and I turn and look at her. She is now quite awake, and look so well as I never saw her since that night at Carfax when we first enter the Count's house. I am amaze, and not at ease then. But she is so bright and tender and thoughtful for me that I forget all fear. I light a fire, for we have brought supply of wood with us, and she prepare food while I undo the horses and set them, tethered in shelter, to feed. Then when I return to the fire she have my supper ready. I go to help her, but she smile, and tell me that she have eat already. That she was so hungry that she would not wait. I like it not, and I have grave doubts. But I fear to affright her, and so I am silent of it. She help me and I eat alone, and then we wrap in fur and lie beside the fire, and I tell her to sleep while I watch. But presently I forget all of watching. And when I sudden remember that I watch, I find her lying quiet, but awake, and looking at me with so bright eyes. Once, twice more the same occur, and I get much sleep till before morning. When I wake I try to hypnotize her, but alas! Though she shut her eyes obedient, she may not sleep. The sun rise up, and up, and up, and then sleep come to her too late, but so heavy that she will not wake. I have to lift her up, and place her sleeping in the carriage when I have harnessed the horses and made all ready. Madam still sleep, and she look in her sleep more healthy and more redder than before. And I like it not. And I am afraid, afraid, afraid! I am afraid of all things, even to think but I must go on my way. The stake we play for is life and death, or more than these, and we must not flinch. 

5 November, morning.--Let me be accurate in everything, for though you and I have seen some strange things together, you may at the first think that I, Van Helsing, am mad. That the many horrors and the so long strain on nerves has at the last turn my brain. 

All yesterday we travel, always getting closer to the mountains, and moving into a more and more wild and desert land. There are great, frowning precipices and much falling water, and Nature seem to have held sometime her carnival. Madam Mina still sleep and sleep. And though I did have hunger and appeased it, I could not waken her, even for food. I began to fear that the fatal spell of the place was upon her, tainted as she is with that Vampire baptism. "Well," said I to myself, "if it be that she sleep all the day, it shall also be that I do not sleep at night." As we travel on the rough road, for a road of an ancient and imperfect kind there was, I held down my head and slept. 

Again I waked with a sense of guilt and of time passed, and found Madam Mina still sleeping, and the sun low down. But all was indeed changed. The frowning mountains seemed further away, and we were near the top of a steep rising hill, on summit of which was such a castle as Jonathan tell of in his diary. At once I exulted and feared. For now, for good or ill, the end was near. 

I woke Madam Mina, and again tried to hypnotize her, but alas! unavailing till too late. Then, ere the great dark came upon us, for even after down sun the heavens reflected the gone sun on the snow, and all was for a time in a great twilight. I took out the horses and fed them in what shelter I could. Then I make a fire, and near it I make Madam Mina, now awake and more charming than ever, sit comfortable amid her rugs. I got ready food, but she would not eat, simply saying that she had not hunger. I did not press her, knowing her unavailingness. But I myself eat, for I must needs now be strong for all. Then, with the fear on me of what might be, I drew a ring so big for her comfort, round where Madam Mina sat. And over the ring I passed some of the wafer, and I broke it fine so that all was well guarded. She sat still all the time, so still as one dead. And she grew whiter and even whiter till the snow was not more pale, and no word she said. But when I drew near, she clung to me, and I could know that the poor soul shook her from head to feet with a tremor that was pain to feel. 

I said to her presently, when she had grown more quiet, "Will you not come over to the fire?" for I wished to make a test of what she could. She rose obedient, but when she have made a step she stopped, and stood as one stricken. 

"Why not go on?" I asked. She shook her head, and coming back, sat down in her place. Then, looking at me with open eyes, as of one waked from sleep, she said simply,"I cannot!" and remained silent. I rejoiced, for I knew that what she could not, none of those that we dreaded could. Though there might be danger to her body, yet her soul was safe! 

Presently the horses began to scream, and tore at their tethers till I came to them and quieted them. When they did feel my hands on them, they whinnied low as in joy,and licked at my hands and were quiet for a time. Many times through the night did I come to them, till it arrive to the cold hour when all nature is at lowest, and every time my coming was with quiet of them. In the cold hour the fire began to die, and I was about stepping forth to replenish it, for now the snow came in flying sweeps and with it a chill mist. Even in the dark there was a light of some kind, as there ever is over snow, and it seemed as though the snow flurries and the wreaths of mist took shape as of women with trailing garments. All was in dead, grim silence only that the horses whinnied and cowered, as if in terror of the worst. I began to fear, horrible fears. But then came to me the sense of safety in that ring wherein I stood. I began too, to think that my imaginings were of the night, and the gloom, and the unrest that I have gone through, and all the terrible anxiety. It was as though my memories of all Jonathan's horrid experience were befooling me. For the snow flakes and the mist began to wheel and circle round, till I could get as though a shadowy glimpse of those women that would have kissed him. And then the horses cowered lower and lower, and moaned in terror as men do in pain. Even the madness of fright was not to them, so that they could break away. I feared for my dear Madam Mina when these weird figures drew near and circled round. I looked at her, but she sat calm, and smiled at me. When I would have stepped to the fire to replenish it, she caught me and held me back, and whispered, like a voice that one hears in a dream, so low it was. 

"No! No! Do not go without. Here you are safe!" 

I turned to her, and looking in her eyes said, "But you? It is for you that I fear!" 

Whereat she laughed, a laugh low and unreal, and said, "Fear for me! Why fear for me? None safer in all the world from them than I am,"and as I wondered at the meaning of her words, a puff of wind made the flame leap up, and I see the red scar on her forehead. Then, alas! I knew. Did I not, I would soon have learned, for the wheeling figures of mist and snow came closer, but keeping ever without the Holy circle. Then they began to materialize till, if God have not taken away my reason, for I saw it through my eyes. There were before me in actual flesh the same three women that Jonathan saw in the room, when they would have kissed his throat. I knew the swaying round forms, the bright hard eyes, the white teeth, the ruddy color, the voluptuous lips. They smiled ever at poor dear Madam Mina. And as their laugh came through the silence of the night, they twined their arms and pointed to her, and said in those so sweet tingling tones that Jonathan said were of the intolerable sweetness of the water glasses, "Come, sister. Come to us. Come!" 

In fear I turned to my poor Madam Mina, and my heart with gladness leapt like flame. For oh! the terror in her sweet eyes, the repulsion, the horror, told a story to my heart that was all of hope. God be thanked she was not, yet of them. I seized some of the firewood which was by me, and holding out some of the Wafer, advanced on them towards the fire. They drew back before me, and laughed their low horrid laugh. I fed the fire, and feared them not. For I knew that we were safe within the ring, which she could not leave no more than they could enter. The horses had ceased to moan, and lay still on the ground. The snow fell on them softly, and they grew whiter. I knew that there was for the poor beasts no more of terror. 

And so we remained till the red of the dawn began to fall through the snow gloom. I was desolate and afraid, and full of woe and terror. But when that beautiful sun began to climb the horizon life was to me again. At the first coming of the dawn the horrid figures melted in the whirling mist and snow. The wreaths of transparent gloom moved away towards the castle, and were lost. 

Instinctively, with the dawn coming, I turned to Madam Mina, intending to hypnotize her. But she lay in a deep and sudden sleep, from which I could not wake her. I tried to hypnotize through her sleep, but she made no response, none at all, and the day broke. I fear yet to stir. I have made my fire and have seen the horses, they are all dead. Today I have much to do here, and I keep waiting till the sun is up high. For there may be places where I must go, where that sunlight, though snow and mist obscure it, will be to me a safety. 

I will strengthen me with breakfast, and then I will do my terrible work. Madam Mina still sleeps, and God be thanked! She is calm in her sleep . . . 

JONATHAN HARKER'S JOURNAL 

4 November, evening.--The accident to the launch has been a terrible thing for us. Only for it we should have overtaken the boat long ago, and by now my dear Mina would have been free. I fear to think of her, off on the wolds near that horrid place. We have got horses, and we follow on the track. I note this whilst Godalming is getting ready. We have our arms. The Szgany must look out if they mean to fight. Oh, if only Morris and Seward were with us. We must only hope! If I write no more Goodby Mina! God bless and keep you. 

DR. SEWARD'S DIARY 

5 November.--With the dawn we saw the body of Szgany before us dashing away from the river with their leiter wagon. They surrounded it in a cluster, and hurried along as though beset. The snow is falling lightly and there is a strange excitement in the air. It may be our own feelings, but the depression is strange. Far off I hear the howling of wolves. The snow brings them down from the mountains, and there are dangers to all of us, and from all sides. The horses are nearly ready, and we are soon off. We ride to death of some one. God alone knows who, or where, or what, or when, or how it may be . . . 

DR. VAN HELSING'S MEMORANDUM 

5 November, afternoon.--I am at least sane. Thank God for that mercy at all events, though the proving it has been dreadful. When I left Madam Mina sleeping within the Holy circle, I took my way to the castle. The blacksmith hammer which I took in the carriage from Veresti was useful, though the doors were all open I broke them off the rusty hinges, lest some ill intent or ill chance should close them, so that being entered I might not get out. Jonathan's bitter experience served me here. By memory of his diary I found my way to the old chapel, for I knew that here my work lay. The air was oppressive. It seemed as if there was some sulphurous fume, which at times made me dizzy. Either there was a roaring in my ears or I heard afar off the howl of wolves. Then I bethought me of my dear Madam Mina, and I was in terrible plight. The dilemma had me between his horns. 

Her, I had not dare to take into this place, but left safe from the Vampire in that Holy circle. And yet even there would be the wolf! I resolve me that my work lay here, and that as to the wolves we must submit, if it were God's will. At any rate it was only death and freedom beyond. So did I choose for her. Had it but been for myself the choice had been easy, the maw of the wolf were better to rest in than the grave of the Vampire! So I make my choice to go on with my work. 

I knew that there were at least three graves to find, graves that are inhabit. So I search, and search, and I find one of them. She lay in her Vampire sleep, so full of life and voluptuous beauty that I shudder as though I have come to do murder. Ah, I doubt not that in the old time, when such things were, many a man who set forth to do such a task as mine, found at the last his heart fail him, and then his nerve. So he delay, and delay, and delay, till the mere beauty and the fascination of the wanton Undead have hypnotize him. And he remain on and on, till sunset come, and the Vampire sleep be over. Then the beautiful eyes of the fair woman open and look love, and the voluptuous mouth present to a kiss, and the man is weak. And there remain one more victim in the Vampire fold. One more to swell the grim and grisly ranks of the Undead! . . . 

There is some fascination, surely, when I am moved by the mere presence of such an one, even lying as she lay in a tomb fretted with age and heavy with the dust of centuries, though there be that horrid odor such as the lairs of the Count have had. Yes, I was moved. I, Van Helsing, with all my purpose and with my motive for hate. I was moved to a yearning for delay which seemed to paralyze my faculties and to clog my very soul. It may have been that the need of natural sleep, and the strange oppression of the air were beginning to overcome me. Certain it was that I was lapsing into sleep, the open eyed sleep of one who yields to a sweet fascination, when there came through the snow stilled air a long, low wail, so full of woe and pity that it woke me like the sound of a clarion. For it was the voice of my dear Madam Mina that I heard. 

Then I braced myself again to my horrid task, and found by wrenching away tomb tops one other of the sisters, the other dark one. I dared not pause to look on her as I had on her sister, lest once more I should begin to be enthrall. But I go on searching until, presently, I find in a high great tomb as if made to one much beloved that other fair sister which, like Jonathan I had seen to gather herself out of the atoms of the mist. She was so fair to look on, so radiantly beautiful, so exquisitely voluptuous, that the very instinct of man in me, which calls some of my sex to love and to protect one of hers, made my head whirl with new emotion. But God be thanked, that soul wail of my dear Madam Mina had not died out of my ears. And, before the spell could be wrought further upon me, I had nerved myself to my wild work. By this tim e I had searched all the tombs in the chapel, so far as I could tell. And as there had been only three of these Undead phantoms around us in the night, I took it that there were no more of active Undead existent. There was one great tomb more lordly than all the rest. Huge it was, and nobly proportioned. On it was but one word. 

DRACULA 

This then was the Undead home of the King Vampire, to whom so many more were due. Its emptiness spoke eloquent to make certain what I knew. Before I began to restore these women to their dead selves through my awful work, I laid in Dracula's tomb some of the Wafer, and so banished him from it, Undead, for ever. 

Then began my terrible task, and I dreaded it. Had it been but one, it had been easy, comparative. But three! To begin twice more after I had been through a deed of horror. For it was terrible with the sweet Miss Lucy, what would it not be with these strange ones who had survived through centuries, and who had been strenghtened by the passing of the years. Who would, if they could, have fought for their foul lives . . . 

Oh, my friend John, but it was butcher work. Had I not been nerved by thoughts of other dead, and of the living over whom hung such a pall of fear, I could not have gone on. I tremble and tremble even yet, though till all was over, God be thanked, my nerve did stand. Had I not seen the repose in the first place, and the gladness that stole over it just ere the final dissolution came, as realization that the soul had been won, I could not have gone further with my butchery. I could not have endured the horrid screeching as the stake drove home, the plunging of writhing form, and lips of bloody foam. I should have fled in terror and left my work undone. But it is over! And the poor souls, I can pity them now and weep, as I think of them placid each in her full sleep of death for a short moment ere fading. For, friend John, hardly had my knife severed the head of each, before the whole body began to melt away and crumble into its native dust, as though the death that should have come centuries agone had at last assert himself and say at once and loud,"I am here!" 

Before I left the castle I so fixed its entrances that never more can the Count enter there Undead. 

When I stepped into the circle where Madam Mina slept, she woke from her sleep and, seeing me, cried out in pain that I had endured too much. 

"Come!" she said, "come away from this awful place! Let us go to meet my husband who is, I know, coming towards us." She was looking thin and pale and weak. But her eyes were pure and glowed with fervor. I was glad to see her paleness and her illness, for my mind was full of the fresh horror of that ruddy vampire sleep. 

And so with trust and hope, and yet full of fear, we go eastward to meet our friends, and him, whom Madam Mina tell me that she know are coming to meet us. 

MINA HARKER'S JOURNAL 

6 November.--It was late in the afternoon when the Professor and I took our way towards the east whence I knew Jonathan was coming. We did not go fast, though the way was steeply downhill, for w e had to take heavy rugs and wraps with us. We dared not face the possibility of being left without warmth in the cold and the snow. We had to take some of our provisions too, for we were in a perfect desolation, and so far as we could see through the snowfall, there was not even the sign of habitation. When we had gone about a mile, I was tired with the heavy walking and sat down to rest. Then we looked back and saw where the clear line of Dracula's castle cut the sky. For we were so deep under the hill whereon it was set that the angle of perspective of the Carpathian mountains was far below it. We saw it in all its grandeur, perched a thousand feet on the summit of a sheer precipice, and with seemingly a great gap between it and the steep of the adjacent mountain on any side. There was something wild and uncanny about the place. We could hear the distant howling of wolves. They were far off, but the sound, even though coming muffled through the deadening snowfall, was full of terror. I knew from the way Dr. Van Helsing was searching about that he was trying to seek some strategic point, where we would be less exposed in case of attack. The rough roadway still led downwards. We could trace it through the drifted snow. 

In a little while the Professor signalled to me, so I got up and joined him. He had found a wonderful spot, a sort of natural hollow in a rock, with an entrance like a doorway between two boulders. He took me by the hand and drew me in. 

"See!" he said,"here you will be in shelter. And if the wolves do come I can meet them one by one." 

He brought in our furs, and made a snug nest for me, and got out some provisions and forced them upon me. But I could not eat, to even try to do so was repulsive to me, and much as I would have liked to please him, I could not bring myself to the attempt. He looked very sad, but did not reproach me. Taking his field glasses from the case, he stood on the top of the rock, and began to search the horizon. 

Suddenly he called out, "Look! Madam Mina, look!Look!" 

I sprang up and stood beside him on the rock. He handed me his glasses and pointed. The snow was now falling more heavily, and swirled about fiercely, for a high wind was beginning to blow. However, there were times when there were pauses between the snow flurries and I could see a long way round. From the height where we were it was possible to see a great distance. And far off, beyond the white waste of snow, I could see the river lying like a black ribbon in kinks and curls as it wound its way. Straight in front of us and not far off, in fact so near that I wondered we had not noticed before, came a group of mounted men hurrying along. In the midst of them was a cart, a long leiter wagon which swept from side to side, like a dog's tail wagging, with each stern inequality of the road. Outlined against the snow as they were, I could see from the men's clothes that they were peasants or gypsies of some kind. 

On the cart was a great square chest. My heart leaped as I saw it, for I felt that the end was coming. The evening was now drawing close, and well I knew that at sunset the Thing, which was till then imprisoned there, would take new freedom and could in any of many forms elude pursuit. In fear I turned to the Professor. To my consternation, however, he was not there. An instant later, I saw him below me. Round the rock he had drawn a circle, such as we had found shelter in last night. 

When he had completed it he stood beside me again saying, "At least you shall be safe here from him!" He took the glasses from me, and at the next lull of the snow swept the whole space below us. "See,"he said,"they come quickly. They are flogging the horses, and galloping as hard as they can." 

He paused and went on in a hollow voice, "They are racing for the sunset. We may be too late. God's will be done!" Down came another blinding rush of driving snow, and the whole landscape was blotted out. It soon passed, however, and once more his glasses were fixed on the plain. 

Then came a sudden cry, "Look! Look! Look! See, two horsemen follow fast, coming up from the south. It must be Quincey and John. Take the glass. Look before the snow blots it all out!" I took it and looked. The two men might be Dr. Seward and Mr. Morris. I knew at all events that neither of them was Jonathan. At the same time I knew that Jonathan was not far off. Looking around I saw on the north side of the coming party two other men, riding at breakneck speed. One of them I knew was Jonathan, and the other I took, of course, to be Lord Godalming. They too, were pursuing the party with the cart. When I told the Professor he shouted in glee like a schoolboy, and after looking intently till a snow fall made sight impossible, he laid his Winchester rifle ready for use against the boulder at the opening of our shelter. 

"They are all converging," he said."When the time comes we shall have gypsies on all sides." I got out my revolver ready to hand, for whilst we were speaking the howling of wolves came louder and closer. When the snow storm abated a moment we looked again. It was strange to see the snow falling in such heavy flakes close to us, and beyond, the sun shining more and more brightly as it sank down towards the far mountain tops. Sweeping the glass all around us I could see here and there dots moving singly and in twos and threes and larger numbers. The wolves were gathering for their prey. 

Every instant seemed an age whilst we waited. The wind came now in fierce bursts, and the snow was driven with fury as it swept upon us in circling eddies. At times we could not see an arm's length before us. But at others, as the hollow sounding wind swept by us, it seemed to clear the air space around us so that we could see afar off. We had of late been so accustomed to watch for sunrise and sunset, that we knew with fair accuracy when it would be. And we knew that before long the sun would set. It was hard to believe that by our watches it was less than an hour that we waited in that rocky shelter before the various bodies began to converge close upon us. The wind came now with fiercer and more bitter sweeps, and more steadily from the north. It seemingly had driven the snow clouds from us, for with only occasional bursts, the snow fell. We could distinguish clearly the individuals of each party, the pursued and the pursuers. Strangely enough those pursued did not seem to realize, or at least to care, that they were pursued. They seemed, however, to hasten with redoubled speed as the sun dropped lower and lower on the mountain tops. 

Closer and closer they drew. The Professor and I crouched down behind our rock, and held our weapons ready. I could see that he was determined that they should not pass. One and all were quite unaware of our presence. 

All at once two voices shouted out to, "Halt!" One was my Jonathan's, raised in a high key of passion. The other Mr. Morris' strong resolute tone of quiet command. The gypsies may not have known the language, but there was no mistaking the tone, in whatever tongue the words were spoken. Instinctively they reined in, and at the instant Lord Godalming and Jonathan dashed up at one side and Dr. Seward and Mr. Morris on the other. The leader of the gypsies, a splendid looking fellow who sat his horse like a centaur, waved them back, and in a fierce voice gave to his companions some word to proceed. They lashed the horses which sprang forward. But the four men raised their Winchester rifles, and in an unmistakable way commanded them to stop. At the same moment Dr. Van Helsing and I rose behind the rock and pointed our weapons at them. Seeing that they were surrounded the men tightened their reins and drew up. The leader turned to them and gave a word at which every man of the gypsy party drew what weapon he carried, knife or pistol,and held himself in readiness to attack. Issue was joined in an instant. 

The leader, with a quick movement of his rein, threw his horse out in front, and pointed first to the sun, now close down on the hill tops, and then to the castle, said something which I did not understand. For answer, all four men of our party threw themselves from their horses and dashed towards the cart. I should have felt terrible fear at seeing Jonathan in such danger, but that the ardor of battle must have been upon me as well as the rest of them. I felt no fear, but only a wild, surging desire to do something. Seeing the quick movement of our parties, the leader of the gypsies gave a command. His men instantly formed round the cart in a sort of undisciplined endeavor, each one shouldering and pushing the other in his eagerness to carry out the order. 

In the midst of this I could see that Jonathan on one side of the ring of men, and Quincey on the other, were forcing a way to the cart. It was evident that they were bent on finishing their task before the sun should set. Nothing seemed to stop or even to hinder them.Neither the levelled weapons nor the flashing knives of the gypsies in front, nor the howling of the wolves behind, appeared to even attract their attention. Jonathan's impetuosity, and the manifest singleness of his purpose, seemed to overawe those in front of him. Instinctively they cowered aside and let him pass. In an instant he had jumped upon the cart, and with a strength which seemed incredible, raised the great box, and flung it over the wheel to the ground. In the meantime, Mr. Morris had had to use force to pass through his side of the ring of Szgany. All the time I had been breathlessly watching Jonathan I had, with the tail of my eye, seen him pressing desperately forward, and had seen the knives of the gypsies flash as he won a way through them, and they cut at him. He had parried with his great bowie knife, and at first I thought that he too had come through in safety. But as he sprang beside Jonathan, who had by now jumped from the cart, I could see that with his left hand he was clutching at his side, and that the blood was spurting through his fingers. He did not delay notwithstanding this, for as Jonathan, with desperate energy, attacked one end of the chest, attempting to prize off the lid with his great Kukri knife, he attacked the other frantically with his bowie. Under the efforts of both men the lid began to yield. The nails drew with a screeching sound, and the top of the box was thrown back. 

By this time the gypsies, seeing themselves covered by the Winchesters, and at the mercy of Lord Godalming and Dr. Seward, had given in and made no further resistance. The sun was almost down on the mountain tops, and the shadows of the whole group fell upon the snow. I saw the Count lying within the box upon the earth, some of which the rude falling from the cart had scattered over him. He was deathly pale, just like a waxen image, and the red eyes glared with the horrible vindictive look which I knew so well. 

As I looked, the eyes saw the sinking sun, and the look of hate in them turned to triumph. 

But, on the instant, came the sweep and flash of Jonathan's great knife. I shrieked as I saw it shear through the throat. Whilst at the same moment Mr. Morris's bowie knife plunged into the heart. 

It was like a miracle, but before our very eyes, and almost in the drawing of a breath, the whole body crumbled into dust and passed from our sight. 

I shall be glad as long as I live that even in that moment of final dissolution, there was in the face a look of peace, such as I never could have imagined might have rested there. 

The Castle of Dracula now stood out against the red sky, and every stone of its broken battlements was articulated against the light of the setting sun. 

The gypsies, taking us as in some way the cause of the extraordinary disappearance of the dead man, turned, without a word, and rode away as if for their lives. Those who were unmounted jumped upon the leiter wagon and shouted to the horsemen not to desert them. The wolves, which had withdrawn to a safe distance, followed in their wake, leaving us alone. 

Mr. Morris, who had sunk to the ground, leaned on his elbow, holding his hand pressed to his side. The blood still gushed through his fingers. I flew to him, for the Holy circle did not now keep me back, so did the two doctors. Jonathan knelt behind him and the wounded man laid back his head on his shoulder. With a sigh he took, with a feeble effort, my hand in that of his own which was unstained. 

He must have seen the anguish of my heart in my face, for he smiled at me and said, "I am only too happy to have been of service! Oh, God!" he cried suddenly, struggling to a sitting posture and pointing to me. "It was worth for this to die! Look! Look!" 

The sun was now right down upon the mountain top, and the red gleams fell upon my face, so that it was bathed in rosy light. With one impulse the men sank on their knees and a deep and earnest "Amen" broke from all as their eyes followed the pointing of his finger. 

The dying man spoke, "Now God be thanked that all has not been in vain! See! The snow is not more stainless than her forehead! The curse has passed away!" 

And, to our bitter grief, with a smile and in silence, he died, a gallant gentleman. 

NOTE 

Seven years ago we all went through the flames. And the happiness of some of us since then is, we think, well worth the pain we endured. It is an added joy to Mina and to me that our boy's birthday is the same day as that on which Quincey Morris died. His mother holds, I know, the secret belief that some of our brave friend's spirit has passed into him. His bundle of names links all our little band of men together. But we call him Quincey. 

In the summer of this year we made a journey to Transylvania, and went over the old ground which was, and is, to us so full of vivid and terrible memories. It was almost impossible to believe that the things which we had seen with our own eyes and heard with our own ears were living truths. Every trace of all that had been was blotted out. The castle stood as before, reared high above a waste of desolation. 

When we got home we were talking of the old time, which we could all look back on without despair, for Godalming and Seward are both happily married. I took the papers from the safe where they had been ever since our return so long ago. We were struck with the fact, that in all the mass of material of which the record is composed, there is hardly one authentic document. Nothing but a mass of typewriting, except the later notebooks of Mina and Seward and myself, and Van Helsing's memorandum. We could hardly ask any one, even did we wish to, to accept these as proofs of so wild a story. Van Helsing summed it all up as he said, with our boy on his knee. 

"We want no proofs. We ask none to believe us! This boy will some day know what a brave and gallant woman his mother is. Already he knows her sweetness and loving care. Later on he will understand how some men so loved her, that they did dare much for her sake. 

JONATHAN HARKER 
